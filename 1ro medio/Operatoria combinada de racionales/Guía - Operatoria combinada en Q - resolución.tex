\documentclass[spanish,letterpaper, 11pt, addpoints, answers]{exam}
\usepackage[left=2cm, right=2cm, top=2cm, bottom=2.5cm]{geometry}
\PassOptionsToPackage{T1}{fontenc} 
    \usepackage{fontenc} 
    \usepackage[utf8]{inputenc}
    % Cargar babel y configurar para español
    \usepackage[spanish, es-tabla, es-noshorthands]{babel}
    \usepackage{lmodern}
    \usepackage{amsfonts}
    \usepackage{multirow}
    \usepackage{hhline}
    \usepackage[none]{hyphenat}
\usepackage[utf8]{inputenc}
\usepackage{graphics}
\usepackage{color}
\usepackage{amssymb}
\usepackage{amsmath}
\usepackage{enumitem}
\usepackage{xcolor}
\usepackage{cancel}
\usepackage{ragged2e}
\usepackage{graphicx}
\usepackage{multicol}
\usepackage{color}
\usepackage{tikz}
\pointpoints{Punto}{Puntos}

\usepackage{graphicx}
\usepackage{tikz}
\usetikzlibrary{babel,arrows.meta,decorations.pathmorphing, backgrounds,positioning,fit,petri, shapes, shadows}

\usepackage{tikz,color}
\usepackage{pgf-pie} 

\usepackage{stackengine}
\newcommand\xrowht[2][0]{\addstackgap[.5\dimexpr#2\relax]{\vphantom{#1}}}



\CorrectChoiceEmphasis{\color{red}}
\noprintanswers

\renewcommand{\choiceshook}{%
    \setlength{\leftmargin}{30pt}%
}


\everymath={\displaystyle}
\renewcommand{\choicelabel}{\thechoice)}
\renewcommand{\choicelabel}{ 
  \ifnum\value{choice}>0
    \makebox[1.5cm][r]{\raggedleft \thechoice)}
  \else
   \raggedleft \thechoice)
  \fi
}

\setlength{\multicolsep}{0.6em}

\setlength{\columnseprule}{2pt}

\setlength{\columnsep}{2cm}
%%%%%% ---Comment out to add a header image ----

\begin{document}
%\begin{figure}[t]
%\includegraphics[width=1\textwidth,height=1.2\textheight,keepaspectratio]{header-cufm.png}
%\end{figure}

\begin{center}
\textbf{Guía de ejercitación} \\
Operatoria combinada de racionales\\
2do semestre 2023
\end{center}
\extraheadheight{-0.5in}

\runningheadrule \extraheadheight{0.15in}

\vspace{0.15in}
\runningheadrule \extraheadheight{0.14in}

\lhead{\ifcontinuation{Pregunta \ContinuedQuestion\ continua\ldots}{}}
\runningheader{Operatoria combinada en $\mathbb{Q}$}{Guía de ejercitación}{2do semestre 2023}
\runningfooter{}
              {\thepage\ de \numpages}
              {}
\vspace{0.05in}

\nopointsinmargin
\setlength\linefillthickness{0.1pt}
\setlength\answerlinelength{0.1in}
\vspace{0.1in}
%\parbox{6in}{
%\textbf{Objetivo:} Verificar el aprendizaje y comprensión de algunos de los temas fundamentales del nivel. }
%\vspace{0.15in}
\hrule 

\begin{questions}

\question En los siguientes ejercicios de operatoria combinada de números enteros incluye los paréntesis necesarios que no afecten al la prioridad de las operaciones.\\

\begin{itemize}
  \item[a.] $-3\cdot 4+(-15)\div 5\div (-3)-3=$
  \item[b.] $4+3\cdot (-5)\cdot 6-18\div(-3)+1=$
  \item[c.] $-5+2-1\cdot5\div(-1)\cdot(-14)-7+2 =$  
\end{itemize}

\question Calcula las siguientes sumas y restas de racionales.


  

\begin{itemize}
\item[a.]$\dfrac{5}{8}+\dfrac{3}{14}=$
\vspace{3cm}
\item[b.]$\dfrac{13}{4}-\dfrac{5}{2}=$
\vspace{3cm}
\item[c.]$\dfrac{1}{2}-\dfrac{7}{3}=$
\vspace{3cm}
\item[d.]$\dfrac{4}{5}+\left(\dfrac{-5}{6}\right)-\dfrac{-7}{15}=$
\vspace{3cm}
\item[e.]$\dfrac{1}{8}-\dfrac{3}{4}+\left(\dfrac{7}{-2}\right)=$
\vspace{3cm}
\item[f.]$-\dfrac{-5}{4}+\left(\dfrac{6}{3}-\dfrac{2}{5}\right)+1{,}125=$
\vspace{3cm}
\item[g.]$-0{,}6+2{,}1+(-\dfrac{2}{3}-3{,}\bar{6})=$  
\end{itemize}


\question Resuelve la siguiente multiplicaciones y divisiones de números racionales.

  

\begin{itemize}
\item[a.] $\dfrac{-4}{3}\cdot \left(\dfrac{-2}{5}\right)=$
\vspace{3cm}
\item[b.] $5\cdot \dfrac{5}{-4}\cdot \dfrac{1}{15}=$
\vspace{3cm}
\item[c.] $-2{,}2\cdot \left(-\dfrac{-7}{3}\right)\cdot 1{,}\bar{1}=$
\vspace{3cm}
\item[d.] $\dfrac{1}{7}\div \dfrac{-3}{4}=$
\vspace{3cm}
\item[e.] $-3{,}5\div \left(\dfrac{-2}{5}\right)\div \left(-\dfrac{-5}{3}\right)\div 4{,}2=$
\vspace{3cm}
\item[f.] $-3{,}5\div \left(\left(\dfrac{-2}{5}\right)\div \left(-\dfrac{-5}{3}\right)\right)\div 4{,}2=$
\vspace{3cm}
\item[g.] $\dfrac{-2}{6}\cdot 4\div 3{,}5\div \left(\dfrac{-18}{10}\right)\cdot \left(-\dfrac{1}{3}\right)=$
\vspace{3cm}
\item[h.] $-\dfrac{3}{5}\div 3\div 5\dfrac{6}{5}\cdot \left(-2\dfrac{2}{6}\div1{,}1\bar{9}\right)\cdot \left(-\dfrac{-5}{4}\right)=$
\vspace{3cm}
\item[i.] $\dfrac{\dfrac{6}{4}\div 1{,}5}{\dfrac{-3\cdot 4}{-2}}\div\dfrac{-2\cdot \dfrac{5}{4}}{\dfrac{-1}{-3}}=$
\vspace{3cm}
\end{itemize}


\question Considera los siguientes ejercicios extraidos del item anterior y reescribelos de tal manera que todas las divisiones involucradas sean escritas como fracciones.

\begin{itemize}
  \item[a.] $\dfrac{1}{7}\div \dfrac{-3}{4}=$
  \vspace{3cm}
  \item[b.] $-3{,}5\div \left(\dfrac{-2}{5}\right)\div \left(-\dfrac{-5}{3}\right)\div 4{,}2=$
  \vspace{3cm}
  \item[c.] $-3{,}5\div \left(\left(\dfrac{-2}{5}\right)\div \left(-\dfrac{-5}{3}\right)\right)\div 4{,}2=$
  \vspace{3cm}
  \item[d.] $\dfrac{-2}{6}\cdot 4\div 3{,}5\div \left(\dfrac{-18}{10}\right)\cdot \left(-\dfrac{1}{3}\right)=$
  \vspace{3cm}
  \item[e.] $-\dfrac{3}{5}\div 3\div 5\dfrac{6}{5}\cdot \left(-2\dfrac{2}{6}\div1{,}1\bar{9}\right)\cdot \left(-\dfrac{-5}{4}\right)=$
  \vspace{3cm}
  \item[f.] $\dfrac{\dfrac{6}{4}\div 1{,}5}{\dfrac{-3\cdot 4}{-2}}\div\dfrac{-2\cdot \dfrac{5}{4}}{\dfrac{-1}{-3}}=$
  \vspace{3cm}
\end{itemize}

\question Resuelve los siguientes ejercicios de operatoria combinada de números racionales.

\textbf{Nivel 1.}
\begin{itemize}
\item[a.] $\dfrac{3}{4}+\left(\dfrac{-2}{3}\right)\div \frac{5}{9}=$
\vspace{3cm}
\item[b.] $0{,}\bar{2}-\dfrac{1}{7}\cdot 0{,}3\bar{5}=$
\vspace{3cm}
\item[c.] $1{,}\bar{1}-1{,}\bar{2}+1{,}\bar{3}-1{,}\bar{4}=$
\vspace{3cm}
\item[d.] $5{,}5\cdot \left(\dfrac{2}{-15}\right)-\dfrac{6}{4}\div \dfrac{-2}{9}=$
\vspace{3cm}
\item[e.] $-5+\left(-\dfrac{5}{6}\right)\cdot \dfrac{-2}{5}-\left(-\dfrac{-2}{3}\right)\div \left(\dfrac{1}{-2}\right)=$.
\vspace{3cm}
\end{itemize}

\textbf{Nivel 2.}
\begin{itemize}
\item[a.] $\left(-2+\dfrac{2}{5}\right)\div \left(\dfrac{-3}{2}\right)-\dfrac{7}{4}=$.
\vspace{4cm}
\item[b.] $-\left(-\dfrac{3}{4}\cdot \left(-0{,}\bar{5}\right)-\left(-\dfrac{-1}{3}\right)\cdot\left(-\dfrac{2}{3}\right)\right)\div \dfrac{19}{4}-4=$
\vspace{4cm}
\item[c.] $\dfrac{3}{5}\div \dfrac{-2}{4}-\left(\dfrac{7}{4}-\dfrac{1}{-2}\right)\div \dfrac{-6}{5}=$
\vspace{4cm}
\item[d.] $\left(\dfrac{2}{3}-3\right)\div\left(-\dfrac{4}{5}+\dfrac{3}{2}\right)-\dfrac{1}{6}\div \dfrac{-5}{12}=$ 
\vspace{4cm}
\item[e.] $\left(-\dfrac{-4}{-3}\cdot \left(\dfrac{-3}{5}+3{,}\bar{5}\right)\div 0{,}\bar{3}-2\right)\div \left(0{,}5+0{,}\bar{5}\right)=$ 
\vspace{4cm}
\end{itemize}

\textbf{Nivel 3.}

\begin{itemize}
\item[a.] $\dfrac{\dfrac{3}{2}-\dfrac{5}{4}}{1-\dfrac{7}{5}}=$
\vspace{5cm}
\item[b.] $\dfrac{15}{-4}-\dfrac{-1+\left(-\dfrac{3}{2}\right)}{\dfrac{8}{5}\cdot \left(-0{,}2\right)}=$
\vspace{5cm}
\item[c.] $\dfrac{\dfrac{1}{9-\dfrac{5}{3}}}{\dfrac{-2\cdot 5+8}{\dfrac{-2}{5}}}=$
\vspace{5cm}
\item[d.] $\dfrac{-5}{4}\div \dfrac{\dfrac{3}{4}-\dfrac{-1}{5}\div 1{,}\bar{5}}{\dfrac{\dfrac{-2}{3}+3}{0{,}4}}=$
\vspace{5cm}
\item[e.] $1+\dfrac{2}{2-\dfrac{3}{3+\dfrac{4}{4-\dfrac{1}{5}}}}=$
\vspace{5cm}
\end{itemize}

\end{questions}
\end{document}