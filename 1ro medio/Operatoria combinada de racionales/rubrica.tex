\documentclass[11pt, letterpaper]{book}


\usepackage{pgfplots}
\pgfplotsset{compat=1.18}
\usepackage[utf8]{inputenc}
\usepackage[spanish]{babel}
\usepackage{amsmath, amsthm}
\usepackage{amsfonts}
\usepackage{amssymb}
\usepackage{graphicx}
\usepackage[left=2cm,right=2cm,top=2cm,bottom=2cm]{geometry}
\geometry{landscape}
\usepackage[export]{adjustbox}
\usepackage{multirow}
\usepackage{multicol}
\usepackage{setspace}
\usepackage{subfig}
\usepackage{venndiagram}
\usepackage{verbatim}
\usepackage{enumitem}
\usepackage{mdframed}
\usepackage{slashbox}


\usepackage{tikz}
\usetikzlibrary{arrows.meta,bbox}
\usetikzlibrary{decorations.pathreplacing}
\begin{comment}
\tikzset{%
  show curve controls/.style={
    postaction={
      decoration={
        show path construction,
        curveto code={
          \draw [blue] 
            (\tikzinputsegmentfirst) -- (\tikzinputsegmentsupporta)
            (\tikzinputsegmentlast) -- (\tikzinputsegmentsupportb);
          \fill [red, opacity=0.5] 
            (\tikzinputsegmentsupporta) circle [radius=.5ex]
            (\tikzinputsegmentsupportb) circle [radius=.5ex];
        }
      },
      decorate
}}}
\end{comment}
\usepackage{stackengine}
\newcommand\xrowht[2][0]{\addstackgap[.5\dimexpr#2\relax]{\vphantom{#1}}}


\usepackage{graphicx}
\usepackage{tikz}
\usetikzlibrary{babel,arrows.meta,decorations.pathmorphing, backgrounds,positioning,fit,petri, shapes, shadows}

\usepackage{tikz,color}
\usepackage{pgf-pie}
\usepackage{pgfplots} 

\theoremstyle{plain}% default
\newtheorem{teo}{Teorema}[section]
\newtheorem{lem}[teo]{Lema}
\newtheorem{prop}[teo]{Proposición}
\newtheorem*{cor}{Corolario}
\newmdtheoremenv{defi}{Definición}[section]
\newtheorem{conj}{Conjetura}[section]
\newtheorem{propi}{Propiedad}[section]

\theoremstyle{definition}
\newtheorem{ejer}{\textit{Ejercicio}}[section]
\newtheorem{ejem}{\textit{Ejemplo}}[section]
\newtheorem*{sol}{Solución}

\theoremstyle{remark}
\newtheorem{obs}{Observación}[section]
\newtheorem*{nota}{Nota}
\newtheorem{caso}{Caso}
\newtheorem*{tips}{Tips}




\newcommand{\paso}[1]{\hspace{1cm}\linebreak\hspace{1cm}\textit{#1 paso. }}


%%%%%%%%%%%%%%%%%%%%%%%%%%%%%%%%%%%%
%%%%%%%%%%%%%%%%%%%%%%%%%%%%%%%%%%%%
%%%%%%%%%%%%%%%%%%%%%%%%%%%%%%%%%%%%

\makeatletter
\newenvironment{myminipage}[1]%
    {\let\@parboxrestore\relax\begin{minipage}{#1}}%
    {\end{minipage}}
\makeatother
%%%%%%%%%%%%%%%%%%%%%%%%%%%%%%%%%%%%%%
%%%%%%%%%%%%%%%%%%%%%%%%%%%%%%%%%%%%%%

\newcounter{conserva}

\newcounter{question}
\newif\ifinchoices
\inchoicesfalse
\newenvironment{questions}{%
  \list{\thequestion.\hspace{0.6cm}}%
  {%
    \usecounter{question}%
    \def\question{\inchoicesfalse\item}%
    \settowidth{\leftmargin}{10.\hskip\labelsep}%
    \labelwidth\leftmargin\advance\labelwidth-\labelsep
  }%
}
{%
  \endlist
}%

\newcounter{choice}
\renewcommand\thechoice{\Alph{choice}}
\newcommand\choicelabel{\thechoice)}
\def\choice{%
  \ifinchoices
    % Do nothing
  \else
    \startchoices
  \fi
  \refstepcounter{choice}%
  %\ifnum\value{choice}>0\relax
  %\penalty -50\hskip 1em plus 1em\relax
  %\fi
  \ifnum\value{choice}>1{\vspace{-0.2cm}}
  
  \fi
  \choicelabel
  \nobreak
  \enskip
}% choice
\def\CorrectChoice{%
  \choice
  \addanswer{\thequestion}{\thechoice}%
}
\let\correctchoice\CorrectChoice

\newcommand{\startchoices}{%
  \inchoicestrue
  \setcounter{choice}{0}%
  \par % Uncomment this to have choices always start a new line
  % If we're continuing the paragraph containing the question,
  % then leave a bit of space before the first choice:
  \ifvmode\else\enskip\fi
}%

\newbox\allanswers
\setbox\allanswers=\hbox{}
\newcommand{\addanswer}[2]{%
  \global\setbox\allanswers=\hbox{\unhbox\allanswers #1.~#2\quad}%
}
\newcommand{\showanswers}{%
  \vfill
  \begin{center}
    Alternativas correctas
  \end{center}
  \noindent\unhbox\allanswers
}

%%%%%%%%%%%%%%%%%%%%%%%%%%%%%%%%%%%%
%%%%%%%%%%%%%%%%%%%%%%%%%%%%%%%%%%%%
%%%%%%%%%%%%%%%%%%%%%%%%%%%%%%%%%%%%










\author{Mauro Díaz}
\title{Apuntes\\Probabilidad y estadística descriptiva e inferencial}

\begin{document}

\begin{tabular}{|p{0.2\textwidth}|p{0.2\textwidth}|p{0.2\textwidth}|p{0.2\textwidth}|}\hline
    \textbf{0 puntos}&\textbf{1 punto}&\textbf{2 puntos}&\textbf{3 puntos}\\ \hline
    No responde.&Cumple con la prioridad de las operaciones y los símbolos de agrupación, y ejecuta el precedimiento para sumar, restar y multiplicar fracciones, sin embargo la división la realiza de manera erronea.&Ejecuta de manera correcta el algoritmo de la suma, resta, multiplicación y división, además de respetar la prioridad de las operaciones, sin embargo acumula más de 2 a 5 errores en la operatoria de números enteros.&Ejecuta de manera correcta el algoritmo de la suma, resta, multiplicación y división, además de respetar la prioridad de las operaciones, sin embargo acumula, como máximo, 1 error en la operatoria de números enteros.\\ \hline 

    Solo transcribe el ejercicio, sin cambios. &Ejecuta de manera correcta el algoritmo de la suma, resta, multiplicación y división de racionales, sin embargo no cumple con la prioridad de las operaciones y los símbolos de agrupación.&&\\ \hline

    Comete el error al sumar o resta fracciones sumando o restando numeradores y denemoniadores, es decir, $$\frac{a}{b}+\frac{c}{d}=\frac{a+c}{b+d}$$.&Ejecuta de manera correcta el algoritmo de la suma, resta, multiplicación y división, además de respetar la prioridad de las operaciones, sin embargo acumula más de 5 errores en la operatoria de números enteros.&&\\ \hline

    Comete error al multiplicar fracciones al igualar denominadores y luego multiplicar los numeradores y mantener intacto de denominador, es decir, $$\frac{a}{c}\cdot \frac{b}{c}=\frac{a\cdot b}{c}$$&En la suma y resta de racionales, si bien busca igualar denominadores, durante la amplificación o simplificación solo opera con el denominador y omite el numerado, o bien opera por numeros diferentes.&&\\ \hline


    
\end{tabular}


\end{document}