\documentclass[spanish,letterpaper, 11pt, addpoints, answers]{exam}
\usepackage[left=2cm, right=2cm, top=2cm, bottom=2.5cm]{geometry}
\PassOptionsToPackage{T1}{fontenc} 
    \usepackage{fontenc} 
    \usepackage[utf8]{inputenc}
    % Cargar babel y configurar para español
    \usepackage[spanish, es-tabla, es-noshorthands]{babel}
    \usepackage{lmodern}
    \usepackage{amsfonts}
    \usepackage{multirow}
    \usepackage{hhline}
    \usepackage[none]{hyphenat}
\usepackage[utf8]{inputenc}
\usepackage{graphics}
\usepackage{color}
\usepackage{amssymb}
\usepackage{amsmath}
\usepackage{enumitem}
\usepackage{xcolor}
\usepackage{cancel}
\usepackage{ragged2e}
\usepackage{graphicx}
\usepackage{multicol}
\usepackage{color}
\usepackage{tikz}
\pointpoints{Punto}{Puntos}

\usepackage{graphicx}
\usepackage{tikz}
\usetikzlibrary{babel,arrows.meta,decorations.pathmorphing, backgrounds,positioning,fit,petri, shapes, shadows}

\usepackage{tikz,color}
\usepackage{pgf-pie} 

\usepackage{stackengine}
\newcommand\xrowht[2][0]{\addstackgap[.5\dimexpr#2\relax]{\vphantom{#1}}}



\CorrectChoiceEmphasis{\color{red}}
\noprintanswers

\renewcommand{\choiceshook}{%
    \setlength{\leftmargin}{30pt}%
}


\everymath={\displaystyle}
\renewcommand{\choicelabel}{\thechoice)}
\renewcommand{\choicelabel}{ 
  \ifnum\value{choice}>0
    \makebox[1.5cm][r]{\raggedleft \thechoice)}
  \else
   \raggedleft \thechoice)
  \fi
}

\setlength{\multicolsep}{0.6em}

\setlength{\columnseprule}{2pt}

\setlength{\columnsep}{2cm}
%%%%%% ---Comment out to add a header image ----

\begin{document}
%\begin{figure}[t]
%\includegraphics[width=1\textwidth,height=1.2\textheight,keepaspectratio]{header-cufm.png}
%\end{figure}

\begin{center}
\textbf{Guía de ejercitación} \\
Números racionales\\
1er semestre 2024
\end{center}
\extraheadheight{-0.5in}

\runningheadrule \extraheadheight{0.15in}

\vspace{0.15in}
\runningheadrule \extraheadheight{0.14in}

\lhead{\ifcontinuation{Pregunta \ContinuedQuestion\ continua\ldots}{}}
\runningheader{Números racionales}{Guía de ejercitación}{1er semestre 2024}
\runningfooter{}
              {\thepage\ de \numpages}
              {}
\vspace{0.05in}

\nopointsinmargin
\setlength\linefillthickness{0.1pt}
\setlength\answerlinelength{0.1in}
\vspace{0.1in}
%\parbox{6in}{
%\textbf{Objetivo:} Verificar el aprendizaje y comprensión de algunos de los temas fundamentales del nivel. }
%\vspace{0.15in}
\hrule 

\begin{questions}

\question Determina el mínimo común multiplo y el máximo comun divisor de los siguientes grupos de números naturales.

\begin{itemize}
  \item $14,18$
  \item $6,12,28$
  \item $45,27,63$
  \item $8,6,15,9$
  \item $162,108,135$
\end{itemize}

\question Amplifica las siguientes fracciones para cumplir con lo pedido.

  \begin{itemize}
  \item[a.]Amplifica $\dfrac{3}{7}$ para que el denominador sea 21.
  \item[b.]Amplifica $\dfrac{5}{4}$ para que el denominador sea 36.
  \item[c.]Amplifica $\dfrac{7}{15}$ para que el denominador sea 45.
  \item[d.]Amplifica $\dfrac{13}{6}$ y $\dfrac{5}{9}$ para que sus denominadores sean iguales.
  \item[e.]Amplifica $\dfrac{-1}{4}$ y $\dfrac{3}{10}$ para que sus denominadores sean iguales.
  \end{itemize}

  \question Simplifica las siguientes fracciones hasta obtener una fracción irreducible.
  
  \begin{itemize}
  \item[a.]$\dfrac{48}{30}=$
  \item[b.]$\dfrac{64}{80}=$
  \item[c.]$\dfrac{90}{378}=$
  \item[d.]$\dfrac{21}{77}=$
  \item[e.]$\dfrac{125}{175}=$
  \end{itemize}

  \question Calcula las siguientes multiplicaciónes y divisiones de números enteros

  \begin{itemize}
  \item[a. ]$8\cdot (-9) =$
  \item[b. ]$-12\cdot 5\div (-6) =$
  \item[c. ]$(-1)\cdot 3\cdot (-6)\cdot (-5)\cdot (-4)=$
  \item[d. ]$(-15)\cdot 24\div(20)\cdot (-9)\div (-6) =$
  \item[e. ]$\underbrace{(-1)\cdot (-1)\cdot (-1)\cdot \ldots \cdot (-1)}_{\text{12638 veces}} =$
  \end{itemize}
  \newpage
  \question Resuelve la siguiente operatoria combinada de números enteros.
  
  \textbf{Nivel 1.}
  \begin{itemize}
  \item[a.] $6\cdot(-3)+5=$.
  
  \item[b.] $-27\div (-3)\cdot 3+(-3)=$
  
  \item[c.] $-13+4\cdot (-3)-15\div (5)=$
  
  \item[d.] $-1-1-1\cdot(-1)+1=$
  
  \item[e.] $2+(-2)\div (2)+5\cdot (-2)=$
  \end{itemize}
  
  \textbf{Nivel 2.}
  \begin{itemize}
  \item[a.] $-5+(-3)\cdot 3-(-12)\div (-2)=$.
  
  \item[b.] $7\cdot 4\cdot (-6)\div (-12)+(-6)\div 2\cdot (-3)=$
  
  \item[c.] $-1\cdot (-1)+1\cdot (-1)\div (-1)-(-1)\cdot (-1)-1=$
  
  \item[d.] $(-2)\cdot (-5)\div (-2)-(-18)\div(-3)\cdot (-2)+(-12)=$
  
  \item[e.] $(-6)\cdot(-5)+6\div(-6)-(-4)=$
  
  \end{itemize}
  
  \textbf{Nivel 3.}
  \begin{itemize}
  \item[a.] $-7+(-3\cdot (-5)+1)\div(-4)=$.
  
  \item[b.] $-(-6\cdot (-7)+6\cdot(-3))\div 8-4=$
  
  \item[c.] $(-8\cdot (-12))\div (-5+(-3))+(-4\cdot 3)=$
  
  \item[d.] $2+(9-6-15-3)\div (5+(-7)-13+9-(-1))=$ 
  
  \item[e.] $-(-15\cdot (-18)-2\cdot 7)\div (-2)\div (-8)-16=$ 
  
  \end{itemize}
  
  \textbf{Nivel 4.}
  \begin{itemize}
  \item[a.] $(6-(-4\cdot 6)\div 8)\div 3+(-7)\cdot 3=$
  
  \item[b.] $(5\cdot (-11+(-7)))\cdot(-3+5\cdot 2)\div((-3\cdot 5+6)\div (-2-1)+11)=$
  
  \item[c.] $((-2)\cdot (-5+(-8)))\div (-2)-(-18)\div(-3\cdot (-2)+(-12))=$
  
  \item[d.] $((-5+(-7)\cdot 8)+5)\div (-7)-(-3\cdot (-12 )\div (-9))=$
  
  \item[e.] $-(-(8-6\cdot (-4))+(-3))\div ((-5\cdot 3-7)\div(-2)+(-4))=$
  
  \end{itemize}

\question Calcula las suiguientes sumas y restas de fracciones de igual denominador.

\begin{itemize}
  \item[a.] $\dfrac{4}{5}+\dfrac{2}{5}=$
  \item[b.] $\dfrac{-3}{6}+\dfrac{-8}{6}=$
  \item[c.] $\dfrac{-19}{7}-\dfrac{7}{7}=$
  \item[d.] $\dfrac{-1}{8}+\dfrac{-9}{8}+\dfrac{3}{8}=$
  \item[e.] $\dfrac{7}{3}-\dfrac{-2}{3}+\dfrac{1}{3}=$  
\end{itemize}
\newpage
\question Escribe las siguientes fracciones en su expanción decimal.

\begin{multicols}{2}
\begin{itemize}
  \item[a.] $\dfrac{5}{8}=$
  \item[b.] $\dfrac{2}{10}=$
  \item[c.] $\dfrac{1}{4}=$
  \item[d.] $\dfrac{-5}{6}=$
  \item[e.] $\dfrac{4}{7}=$
  \item[f.] $\dfrac{-2}{11}=$
  \item[g.] $\dfrac{13}{5}=$
  \item[h.] $\dfrac{14}{9}=$
  \item[i.] $\dfrac{7}{15}=$
  \item[j.] $\dfrac{-1}{12}=$  
\end{itemize}
\end{multicols}

\question Escribe los siguientes números decimales en su notación fraccionaria.

\begin{multicols}{2}
\begin{itemize}
  \item[a.] $0,3=$
  \item[b.] $12,2=$
  \item[c.] $0,35=$
  \item[d.] $9,9=$
  \item[e.] $123,125=$
  \item[f.] $1,\overline{6}=$
  \item[g.] $0,4\overline{5}=$
  \item[h.] $2,\overline{9}=$
  \item[i.] $25,25\overline{5}=$
  \item[j.] $0,\overline{12}=$  
\end{itemize}
\end{multicols}

\question Calcula las siguientes sumas y restas de racionales.

\begin{multicols}{2}
  

\begin{itemize}
\item[a.]$\dfrac{5}{8}+\dfrac{3}{14}=$
\item[b.]$\dfrac{13}{4}-\dfrac{5}{2}=$
\item[c.]$\dfrac{1}{2}-\dfrac{7}{3}=$
\item[d.]$\dfrac{4}{5}+\left(\dfrac{-5}{6}\right)-\dfrac{-7}{15}=$
\item[e.]$\dfrac{1}{8}-\dfrac{3}{4}+\left(\dfrac{7}{-2}\right)=$
\item[f.]$-\dfrac{-5}{4}+\left(\dfrac{6}{3}-\dfrac{2}{5}\right)+1{,}125=$
\item[g.]$-0{,}6+2{,}1+(-\dfrac{2}{3}-3{,}\bar{6})=$  
\end{itemize}
\end{multicols}

\question Resuelve la siguiente multiplicaciones y divisiones de números racionales.
\begin{multicols}{2}
  

\begin{itemize}
\item[a.] $\dfrac{-4}{3}\cdot \left(\dfrac{-2}{5}\right)=$

\item[b.] $5\cdot \dfrac{5}{-4}\cdot \dfrac{1}{15}=$

\item[c.] $-2{,}2\cdot \left(-\dfrac{-7}{3}\right)\cdot 1{,}\bar{1}=$

\item[d.] $\dfrac{1}{7}\div \dfrac{-3}{4}=$

\item[e.] $-3{,}5\div \left(\dfrac{-2}{5}\right)\div \left(-\dfrac{-5}{3}\right)\div 4{,}2=$

\item[f.] $-3{,}5\div \left(\left(\dfrac{-2}{5}\right)\div \left(-\dfrac{-5}{3}\right)\right)\div 4{,}2=$

\item[g.] $\dfrac{-2}{6}\cdot 4\div 3{,}5\div \left(\dfrac{-18}{10}\right)\cdot \left(-\dfrac{1}{3}\right)=$

\item[h.] $-\dfrac{3}{5}\div 3\div 5\dfrac{6}{5}\cdot \left(-2\dfrac{2}{6}\div1{,}1\bar{9}\right)\cdot \left(-\dfrac{-5}{4}\right)=$

\item[i.] $\dfrac{\dfrac{6}{4}\div 1{,}5}{\dfrac{-3\cdot 4}{-2}}\div\dfrac{-2\cdot \dfrac{5}{4}}{\dfrac{-1}{-3}}=$

\end{itemize}

\end{multicols}
\question Considera los siguientes ejercicios extraidos del item anterior y reescribelos de tal manera que todas las divisiones involucradas sean escritas como fracciones.

\begin{itemize}
  \item[a.] $\dfrac{1}{7}\div \dfrac{-3}{4}=$

  \item[b.] $-3{,}5\div \left(\dfrac{-2}{5}\right)\div \left(-\dfrac{-5}{3}\right)\div 4{,}2=$

  \item[c.] $-3{,}5\div \left(\left(\dfrac{-2}{5}\right)\div \left(-\dfrac{-5}{3}\right)\right)\div 4{,}2=$

  \item[d.] $\dfrac{-2}{6}\cdot 4\div 3{,}5\div \left(\dfrac{-18}{10}\right)\cdot \left(-\dfrac{1}{3}\right)=$

  \item[e.] $-\dfrac{3}{5}\div 3\div 5\dfrac{6}{5}\cdot \left(-2\dfrac{2}{6}\div1{,}1\bar{9}\right)\cdot \left(-\dfrac{-5}{4}\right)=$

  \item[f.] $\dfrac{\dfrac{6}{4}\div 1{,}5}{\dfrac{-3\cdot 4}{-2}}\div\dfrac{-2\cdot \dfrac{5}{4}}{\dfrac{-1}{-3}}=$

\end{itemize}

\question Resuelve los siguientes ejercicios de operatoria combinada de números racionales.

\textbf{Nivel 1.}
\begin{itemize}
\item[a.] $\dfrac{3}{4}+\left(\dfrac{-2}{3}\right)\div \frac{5}{9}=$

\item[b.] $0{,}\bar{2}-\dfrac{1}{7}\cdot 0{,}3\bar{5}=$

\item[c.] $1{,}\bar{1}-1{,}\bar{2}+1{,}\bar{3}-1{,}\bar{4}=$

\item[d.] $5{,}5\cdot \left(\dfrac{2}{-15}\right)-\dfrac{6}{4}\div \dfrac{-2}{9}=$

\item[e.] $-5+\left(-\dfrac{5}{6}\right)\cdot \dfrac{-2}{5}-\left(-\dfrac{-2}{3}\right)\div \left(\dfrac{1}{-2}\right)=$.

\end{itemize}

\textbf{Nivel 2.}
\begin{itemize}
\item[a.] $\left(-2+\dfrac{2}{5}\right)\div \left(\dfrac{-3}{2}\right)-\dfrac{7}{4}=$.

\item[b.] $-\left(-\dfrac{3}{4}\cdot \left(-0{,}\bar{5}\right)-\left(-\dfrac{-1}{3}\right)\cdot\left(-\dfrac{2}{3}\right)\right)\div \dfrac{19}{4}-4=$

\item[c.] $\dfrac{3}{5}\div \dfrac{-2}{4}-\left(\dfrac{7}{4}-\dfrac{1}{-2}\right)\div \dfrac{-6}{5}=$

\item[d.] $\left(\dfrac{2}{3}-3\right)\div\left(-\dfrac{4}{5}+\dfrac{3}{2}\right)-\dfrac{1}{6}\div \dfrac{-5}{12}=$ 

\item[e.] $\left(-\dfrac{-4}{-3}\cdot \left(\dfrac{-3}{5}+3{,}\bar{5}\right)\div 0{,}\bar{3}-2\right)\div \left(0{,}5+0{,}\bar{5}\right)=$ 

\end{itemize}

\textbf{Nivel 3.}
\begin{multicols}{2}
\begin{itemize}
\item[a.] $\dfrac{\dfrac{3}{2}-\dfrac{5}{4}}{1-\dfrac{7}{5}}=$
\vspace{1cm}
\item[b.] $\dfrac{15}{-4}-\dfrac{-1+\left(-\dfrac{3}{2}\right)}{\dfrac{8}{5}\cdot \left(-0{,}2\right)}=$
\vspace{1cm}
\item[c.] $\dfrac{\dfrac{1}{9-\dfrac{5}{3}}}{\dfrac{-2\cdot 5+8}{\dfrac{-2}{5}}}=$
\vspace{1cm}
\item[d.] $\dfrac{-5}{4}\div \dfrac{\dfrac{3}{4}-\dfrac{-1}{5}\div 1{,}\bar{5}}{\dfrac{\dfrac{-2}{3}+3}{0{,}4}}=$
\vspace{1cm}
\item[e.] $1+\dfrac{2}{2-\dfrac{3}{3+\dfrac{4}{4-\dfrac{1}{5}}}}=$

\end{itemize}
\end{multicols}
\end{questions}
\end{document}