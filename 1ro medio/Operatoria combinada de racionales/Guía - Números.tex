\documentclass[spanish,letterpaper, 11pt, addpoints, answers]{exam}
\usepackage[left=2cm, right=2cm, top=2cm, bottom=2.5cm]{geometry}
\PassOptionsToPackage{T1}{fontenc} 
    \usepackage{fontenc} 
    \usepackage[utf8]{inputenc}
    % Cargar babel y configurar para español
    \usepackage[spanish, es-tabla, es-noshorthands]{babel}
    \usepackage{lmodern}
    \usepackage{amsfonts}
    \usepackage{multirow}
    \usepackage{hhline}
    \usepackage[none]{hyphenat}
\usepackage[utf8]{inputenc}
\usepackage{graphics}
\usepackage{color}
\usepackage{amssymb}
\usepackage{amsmath}
\usepackage{enumitem}
\usepackage{xcolor}
\usepackage{cancel}
\usepackage{ragged2e}
\usepackage{graphicx}
\usepackage{multicol}
\usepackage{color}
\usepackage{tikz}
\pointpoints{Punto}{Puntos}

\usepackage{graphicx}
\usepackage{tikz}
\usetikzlibrary{babel,arrows.meta,decorations.pathmorphing, backgrounds,positioning,fit,petri, shapes, shadows}

\usepackage{tikz,color}
\usepackage{pgf-pie} 

\usepackage{stackengine}
\newcommand\xrowht[2][0]{\addstackgap[.5\dimexpr#2\relax]{\vphantom{#1}}}



\CorrectChoiceEmphasis{\color{red}}
\noprintanswers

\renewcommand{\choiceshook}{%
    \setlength{\leftmargin}{30pt}%
}


\everymath={\displaystyle}
\renewcommand{\choicelabel}{\thechoice)}
\renewcommand{\choicelabel}{ 
  \ifnum\value{choice}>0
    \makebox[1.5cm][r]{\raggedleft \thechoice)}
  \else
   \raggedleft \thechoice)
  \fi
}

\setlength{\multicolsep}{0.6em}

\setlength{\columnseprule}{2pt}

\setlength{\columnsep}{2cm}
%%%%%% ---Comment out to add a header image ----

\begin{document}
%\begin{figure}[t]
%\includegraphics[width=1\textwidth,height=1.2\textheight,keepaspectratio]{header-cufm.png}
%\end{figure}

\begin{center}
\textbf{Números racionales y potencias} \\
Guía teórica\\
1er semestre 2024
\end{center}
\extraheadheight{-0.5in}

\runningheadrule \extraheadheight{0.15in}

\vspace{0.15in}
\runningheadrule \extraheadheight{0.14in}

\lhead{\ifcontinuation{Pregunta \ContinuedQuestion\ continua\ldots}{}}
\runningheader{Números racionales y potencias}{Guía teórica}{1er semestre 2024}
\runningfooter{}
              {\thepage\ de \numpages}
              {}
\vspace{0.05in}

\nopointsinmargin
\setlength\linefillthickness{0.1pt}
\setlength\answerlinelength{0.1in}
\vspace{0.1in}
%\parbox{6in}{
%\textbf{Objetivo:} Verificar el aprendizaje y comprensión de algunos de los temas fundamentales del nivel. }
%\vspace{0.15in}
\hrule 

\vspace{0.1in}
\parbox{6in}{
\textbf{Tema:} Operatoria con números enteros}
\vspace{0.15in}
\hrule

\begin{itemize}
  \item Adición en $\mathbb{Z}$: De iguales signos, se suman los valores absolutos y se conserva el signo. De distintos signos, se restan los valores absolutos y se conserva el signo del número de mayor valor absoluto.
  \item Sustracción en $\mathbb{Z}$: Se transforma en una adición y se procede como corresponde, considerando si los valores son de iguales o distintos signos.
  \item Multiplicación y división en $\mathbb{Z}$: De iguales signos, se multiplican o dividen los valores absolutos y el producto siempre es positivo. De distintos signos, se multiplican o dividen los valores absolutos y el producto siempre es negativo.
\end{itemize}

\parbox{6in}{
\textbf{Actividades propuestas}}
\vspace{0.15in}
\hrule 

\begin{questions}

\question Resuelve las adiciones y sustracciones.
\begin{multicols}{2}

\begin{itemize}
  \item[a.] $15+(-8)$
  \item[b.] $-41+(-32)$
  \item[c.] $-(-52)+25$
  \item[d.] $53+(-14)$
  \item[e.] $87-(-21)$
  \item[f.] $345+768-450$
  \item[g.] $45-(234-12)$
  \item[h.] $-(48-43+340)$
  \item[i.] $254-4.898+3.740$
  \item[j.] $-(-478-1.243+148)$
  \item[k.] $5+16+(-4)+25-45$
  \item[l.] $21-231-14+231+14$
\end{itemize}
\end{multicols}

\question Resuelve las multiplicaciones y divisiones.
\begin{multicols}{2}
  

  \begin{itemize}
    \item[a.] $12\cdot(-12)$
    \item[b.] $-(-42)\div 14$
    \item[c.] $72\div (-18)$
    \item[d.] $810\div (-90)$
    \item[e.] $12\cdot 76\div 3$
    \item[f.] $750\div(25\cdot 10)$
    \item[g.] $-(48\div 16\cdot 925)$
    \item[h.] $-(-7\cdot 1.199\cdot 2)$
    \item[i.] $35\div (-7)\cdot 4\div (-20)\cdot (-1)$
    \item[j.] $15\cdot 15\div (-25)\cdot 25\div 45$
    \item[k.] $289\div 17\cdot 17\cdot (-1)\div 17$
    \item[l.] $-66\div (-11)\cdot (-112)\div 12\cdot 10$
  \end{itemize}
\end{multicols}

  \question Resuelve las operaciones combinadas.

  \begin{multicols}{2}
  \begin{itemize}
  \item[a.]$2\cdot \left( 12\div (-6)+(-12)\cdot 6-48\div 16\right)$
  \item[b.]$-1-15+\left((-21)-21\cdot 3\right) +8\cdot 12$
  \item[c.]$\left( 19-120\div (-10)\cdot 6\cdot 10-10\right) \div 9$
  \item[d.]$7+\left( 200\div (-2)+(-9)\cdot 6\div 3+12\right)$
  \item[e.]$18-\left( 8+256\div 16\cdot 4-7\right) \cdot(-2)$
  \item[f.] $-100\div \left( 9\cdot 23\div 3- 45+1\right) \cdot 2$
  \item[g.] $8 + \left( 312\div 12\cdot 6\div  4-1\right) \cdot 2$
  \item[h.] $4\cdot \left( -2-205\div \left(65\div \left(-24+37\right)\right)\right) \cdot (-5)$
  \end{itemize}

\end{multicols}

  \question Resuelve los problemas

  \begin{itemize}
  \item[a. ] Una persona que está en el décimo quinto piso de un edificio, sube 5 pisos, luego baja 11 pisos y finalmente, sube otra cantidad incierta de pisos. Si desde esta última parada, desciendo al primer piso, pasando por 18 pisos, y sale del edificio, ¿cuántos pisos subió en la última parada?
  \item[b. ] Una deuda de \$18.000 se incrementó en \$4.000 cada 30 días, durante 180 dias. Luego de este periodo el monto total adeudado fue dividido en siete cuotas iguales y sin más intereses. ¿Cuánto dinero se debe pagar en cada cuota? ¿Qué número entero representa el monto total de la deuda?
  \end{itemize}
  \newpage

  \parbox{6in}{
    \textbf{Tema:} Operatoria con fracciones}
    \vspace{0.15in}
    \hrule
    
    \begin{itemize}
      \item Adición de fracciones: 
      $$\dfrac{a}{b}+\dfrac{c}{d}=\dfrac{a\cdot d+b\cdot c}{b\cdot d}$$
      \item Sustracción de fracciones: 
      $$\dfrac{a}{b}-\dfrac{c}{d}=\dfrac{a\cdot d-b\cdot c}{b\cdot d}$$
      \item Multiplicación de fracciones:
      $$\dfrac{a}{b}\cdot \dfrac{c}{d}=\dfrac{a\cdot c}{b\cdot d}$$
      \item División de fracciones:
      $$\dfrac{a}{b}\div \dfrac{c}{d}=\dfrac{a}{b}\cdot \dfrac{d}{c}=\dfrac{a\cdot d}{b\cdot c}$$
      \item Amplificar una fracción consiste en multiplicar su numerador y su denominador por un mismo valor.
      \item Simplificar una fracción consiste en dividir su numerador y su denominador por un mismo valor.
    \end{itemize}

    \parbox{6in}{
      \textbf{Actividades propuestas}}
      \vspace{0.15in}
      \hrule 

  \question Resuelve las operaciones entre fracciones. Expresa el resultado como fracción irreductible.
  \begin{multicols}{2}
    
  

  \begin{itemize}
    \item[a.] $\dfrac{3}{2}+\dfrac{5}{3}$
  \item[b.] $\dfrac{8}{9}-\dfrac{1}{2}$
  \item[c.] $\dfrac{12}{17}\cdot \dfrac{5}{6}$
  \item[d.] $\dfrac{8}{15}\div \dfrac{2}{5}$
  \item[e.] $\dfrac{7}{18}-\dfrac{4}{5}+\dfrac{5}{9}$
  \item[f.] $\dfrac{6}{7}\cdot\dfrac{7}{12}\div \dfrac{1}{2}$
  \item[g.] $\dfrac{2}{7}-\dfrac{1}{25}\div \dfrac{2}{3}$
  \item[h.] $\dfrac{21}{25}+\dfrac{24}{25}\cdot \dfrac{1}{12}$
  \item[i.] $\dfrac{4}{5}-\left(\dfrac{1}{5}+\dfrac{3}{8}\right)\div \dfrac{1}{2}$
  \item[j.] $\dfrac{12}{13}\cdot \left(\dfrac{5}{6}-\dfrac{2}{3}\right)-\dfrac{4}{26}$
  \item[k.] $\dfrac{5}{24}\cdot \left(\dfrac{3}{7}\div \dfrac{5}{4}\cdot \dfrac{4}{5}\right)+\dfrac{1}{2}$
  \item[l.] $\dfrac{13}{15}-\dfrac{13}{15}\cdot \dfrac{1}{2}\cdot \dfrac{2}{3}+\dfrac{13}{15}$
  \item[m.] $\left(\dfrac{9}{14}-\dfrac{3}{7}\right)\cdot \dfrac{7}{9}\div \left(\dfrac{1}{2}+\dfrac{1}{3}\right)$
  \item[n.] $\dfrac{21}{25}-\dfrac{7}{13}\cdot \left(\left(\dfrac{1}{2}\cdot \dfrac{2}{7}\right)+\dfrac{3}{5}\right)$
  \item[ñ.] $\dfrac{9}{34}\div \left(\dfrac{5}{23}\cdot \left(\dfrac{7}{2}-\dfrac{9}{10}\right)-\dfrac{14}{17}\right)$  
  \end{itemize}
\end{multicols}
  
\question Resulve los problemas.

\begin{itemize}
  \item[a.] La longitud de una pista de atletismo de cierto recinto deportivo es de 800 m. Si un atleta da 9 vueltas completas más tres cuartos de otra y, otro atleta completo 7 vueltas y un octavo de otra, ¿cuántos metros recorrieron entre ambos?
  \item[b.] Si un kilogramo de queso cuesta \$6.200, uno de paltas cuesta \$1.790 y uno de tomates, \$800, ¿cuánto dinero costará comprar un cuarto de kilogramo de queso, dos de paltas y uno de tomates?
\end{itemize}

\newpage

  \parbox{6in}{
    \textbf{Tema:} Operatoria con decimales}
    \vspace{0.15in}
    \hrule
    
    \begin{itemize}
      \item \textbf{Adición y sustracción de números decimales:} Se deben sumar o restar las cifras con igual orden posicional, de menor a mayor orden; es decir, las milésimas con las milésimas, las centésimas con las centésimas, décimas con décimas, unidades con unidades, decenas con decenas, etc. 
      
      \item \textbf{Multiplicación de números decimales:} Una técnica consiste en multiplicar los factores sin considerar la coma decimal, para luego ubicarla en el producto según el total de cifras decimales de ambos factores.
      \item \textbf{División de números decimales:} Una técnica consiste en amplificar, tanto el dividendo como el divisor, de tal manea que la división solo involucre npumeros naturales. Luego, resolver la división de forma tradicional.
    \end{itemize}

    \parbox{6in}{
      \textbf{Actividades propuestas}}
      \vspace{0.15in}
      \hrule 



\question Calcula las suiguientes sumas y restas de fracciones de igual denominador.

\begin{itemize}
  \item[a.] $\dfrac{4}{5}+\dfrac{2}{5}=$
  \item[b.] $\dfrac{-3}{6}+\dfrac{-8}{6}=$
  \item[c.] $\dfrac{-19}{7}-\dfrac{7}{7}=$
  \item[d.] $\dfrac{-1}{8}+\dfrac{-9}{8}+\dfrac{3}{8}=$
  \item[e.] $\dfrac{7}{3}-\dfrac{-2}{3}+\dfrac{1}{3}=$  
\end{itemize}
\newpage
\question Escribe las siguientes fracciones en su expanción decimal.

\begin{multicols}{2}
\begin{itemize}
  \item[a.] $\dfrac{5}{8}=$
  \item[b.] $\dfrac{2}{10}=$
  \item[c.] $\dfrac{1}{4}=$
  \item[d.] $\dfrac{-5}{6}=$
  \item[e.] $\dfrac{4}{7}=$
  \item[f.] $\dfrac{-2}{11}=$
  \item[g.] $\dfrac{13}{5}=$
  \item[h.] $\dfrac{14}{9}=$
  \item[i.] $\dfrac{7}{15}=$
  \item[j.] $\dfrac{-1}{12}=$  
\end{itemize}
\end{multicols}

\question Escribe los siguientes números decimales en su notación fraccionaria.

\begin{multicols}{2}
\begin{itemize}
  \item[a.] $0,3=$
  \item[b.] $12,2=$
  \item[c.] $0,35=$
  \item[d.] $9,9=$
  \item[e.] $123,125=$
  \item[f.] $1,\overline{6}=$
  \item[g.] $0,4\overline{5}=$
  \item[h.] $2,\overline{9}=$
  \item[i.] $25,25\overline{5}=$
  \item[j.] $0,\overline{12}=$  
\end{itemize}
\end{multicols}

\question Calcula las siguientes sumas y restas de racionales.

\begin{multicols}{2}
  

\begin{itemize}
\item[a.]$\dfrac{5}{8}+\dfrac{3}{14}=$
\item[b.]$\dfrac{13}{4}-\dfrac{5}{2}=$
\item[c.]$\dfrac{1}{2}-\dfrac{7}{3}=$
\item[d.]$\dfrac{4}{5}+\left(\dfrac{-5}{6}\right)-\dfrac{-7}{15}=$
\item[e.]$\dfrac{1}{8}-\dfrac{3}{4}+\left(\dfrac{7}{-2}\right)=$
\item[f.]$-\dfrac{-5}{4}+\left(\dfrac{6}{3}-\dfrac{2}{5}\right)+1{,}125=$
\item[g.]$-0{,}6+2{,}1+(-\dfrac{2}{3}-3{,}\bar{6})=$  
\end{itemize}
\end{multicols}

\question Resuelve la siguiente multiplicaciones y divisiones de números racionales.
\begin{multicols}{2}
  

\begin{itemize}
\item[a.] $\dfrac{-4}{3}\cdot \left(\dfrac{-2}{5}\right)=$

\item[b.] $5\cdot \dfrac{5}{-4}\cdot \dfrac{1}{15}=$

\item[c.] $-2{,}2\cdot \left(-\dfrac{-7}{3}\right)\cdot 1{,}\bar{1}=$

\item[d.] $\dfrac{1}{7}\div \dfrac{-3}{4}=$

\item[e.] $-3{,}5\div \left(\dfrac{-2}{5}\right)\div \left(-\dfrac{-5}{3}\right)\div 4{,}2=$

\item[f.] $-3{,}5\div \left(\left(\dfrac{-2}{5}\right)\div \left(-\dfrac{-5}{3}\right)\right)\div 4{,}2=$

\item[g.] $\dfrac{-2}{6}\cdot 4\div 3{,}5\div \left(\dfrac{-18}{10}\right)\cdot \left(-\dfrac{1}{3}\right)=$

\item[h.] $-\dfrac{3}{5}\div 3\div 5\dfrac{6}{5}\cdot \left(-2\dfrac{2}{6}\div1{,}1\bar{9}\right)\cdot \left(-\dfrac{-5}{4}\right)=$

\item[i.] $\dfrac{\dfrac{6}{4}\div 1{,}5}{\dfrac{-3\cdot 4}{-2}}\div\dfrac{-2\cdot \dfrac{5}{4}}{\dfrac{-1}{-3}}=$

\end{itemize}

\end{multicols}
\question Considera los siguientes ejercicios extraidos del item anterior y reescribelos de tal manera que todas las divisiones involucradas sean escritas como fracciones.

\begin{itemize}
  \item[a.] $\dfrac{1}{7}\div \dfrac{-3}{4}=$

  \item[b.] $-3{,}5\div \left(\dfrac{-2}{5}\right)\div \left(-\dfrac{-5}{3}\right)\div 4{,}2=$

  \item[c.] $-3{,}5\div \left(\left(\dfrac{-2}{5}\right)\div \left(-\dfrac{-5}{3}\right)\right)\div 4{,}2=$

  \item[d.] $\dfrac{-2}{6}\cdot 4\div 3{,}5\div \left(\dfrac{-18}{10}\right)\cdot \left(-\dfrac{1}{3}\right)=$

  \item[e.] $-\dfrac{3}{5}\div 3\div 5\dfrac{6}{5}\cdot \left(-2\dfrac{2}{6}\div1{,}1\bar{9}\right)\cdot \left(-\dfrac{-5}{4}\right)=$

  \item[f.] $\dfrac{\dfrac{6}{4}\div 1{,}5}{\dfrac{-3\cdot 4}{-2}}\div\dfrac{-2\cdot \dfrac{5}{4}}{\dfrac{-1}{-3}}=$

\end{itemize}

\question Resuelve los siguientes ejercicios de operatoria combinada de números racionales.

\textbf{Nivel 1.}
\begin{itemize}
\item[a.] $\dfrac{3}{4}+\left(\dfrac{-2}{3}\right)\div \frac{5}{9}=$

\item[b.] $0{,}\bar{2}-\dfrac{1}{7}\cdot 0{,}3\bar{5}=$

\item[c.] $1{,}\bar{1}-1{,}\bar{2}+1{,}\bar{3}-1{,}\bar{4}=$

\item[d.] $5{,}5\cdot \left(\dfrac{2}{-15}\right)-\dfrac{6}{4}\div \dfrac{-2}{9}=$

\item[e.] $-5+\left(-\dfrac{5}{6}\right)\cdot \dfrac{-2}{5}-\left(-\dfrac{-2}{3}\right)\div \left(\dfrac{1}{-2}\right)=$.

\end{itemize}

\textbf{Nivel 2.}
\begin{itemize}
\item[a.] $\left(-2+\dfrac{2}{5}\right)\div \left(\dfrac{-3}{2}\right)-\dfrac{7}{4}=$.

\item[b.] $-\left(-\dfrac{3}{4}\cdot \left(-0{,}\bar{5}\right)-\left(-\dfrac{-1}{3}\right)\cdot\left(-\dfrac{2}{3}\right)\right)\div \dfrac{19}{4}-4=$

\item[c.] $\dfrac{3}{5}\div \dfrac{-2}{4}-\left(\dfrac{7}{4}-\dfrac{1}{-2}\right)\div \dfrac{-6}{5}=$

\item[d.] $\left(\dfrac{2}{3}-3\right)\div\left(-\dfrac{4}{5}+\dfrac{3}{2}\right)-\dfrac{1}{6}\div \dfrac{-5}{12}=$ 

\item[e.] $\left(-\dfrac{-4}{-3}\cdot \left(\dfrac{-3}{5}+3{,}\bar{5}\right)\div 0{,}\bar{3}-2\right)\div \left(0{,}5+0{,}\bar{5}\right)=$ 

\end{itemize}

\textbf{Nivel 3.}
\begin{multicols}{2}
\begin{itemize}
\item[a.] $\dfrac{\dfrac{3}{2}-\dfrac{5}{4}}{1-\dfrac{7}{5}}=$
\vspace{1cm}
\item[b.] $\dfrac{15}{-4}-\dfrac{-1+\left(-\dfrac{3}{2}\right)}{\dfrac{8}{5}\cdot \left(-0{,}2\right)}=$
\vspace{1cm}
\item[c.] $\dfrac{\dfrac{1}{9-\dfrac{5}{3}}}{\dfrac{-2\cdot 5+8}{\dfrac{-2}{5}}}=$
\vspace{1cm}
\item[d.] $\dfrac{-5}{4}\div \dfrac{\dfrac{3}{4}-\dfrac{-1}{5}\div 1{,}\bar{5}}{\dfrac{\dfrac{-2}{3}+3}{0{,}4}}=$
\vspace{1cm}
\item[e.] $1+\dfrac{2}{2-\dfrac{3}{3+\dfrac{4}{4-\dfrac{1}{5}}}}=$

\end{itemize}
\end{multicols}
\end{questions}
\end{document}