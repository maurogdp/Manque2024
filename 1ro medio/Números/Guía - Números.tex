\documentclass[spanish,letterpaper, 11pt, addpoints, answers]{exam}
\usepackage[left=2cm, right=2cm, top=2cm, bottom=2.5cm]{geometry}
\PassOptionsToPackage{T1}{fontenc} 
    \usepackage{fontenc} 
    \usepackage[utf8]{inputenc}
    % Cargar babel y configurar para español
    \usepackage[spanish, es-tabla, es-noshorthands]{babel}
    \usepackage{lmodern}
    \usepackage{amsfonts}
    \usepackage{multirow}
    \usepackage{hhline}
    \usepackage[none]{hyphenat}
\usepackage[utf8]{inputenc}
\usepackage{graphics}
\usepackage{color}
\usepackage{amssymb}
\usepackage{amsmath}
\usepackage{enumitem}
\usepackage{xcolor}
\usepackage{cancel}
\usepackage{ragged2e}
\usepackage{graphicx}
\usepackage{multicol}
\usepackage{color}
\usepackage{tikz}
\pointpoints{Punto}{Puntos}

\usepackage{graphicx}
\usepackage{tikz}
\usetikzlibrary{babel,arrows.meta,decorations.pathmorphing, backgrounds,positioning,fit,petri, shapes, shadows}

\usepackage{tikz,color}
\usepackage{pgf-pie} 

\usepackage{stackengine}
\newcommand\xrowht[2][0]{\addstackgap[.5\dimexpr#2\relax]{\vphantom{#1}}}



\CorrectChoiceEmphasis{\color{red}}
\noprintanswers

\renewcommand{\choiceshook}{%
    \setlength{\leftmargin}{30pt}%
}


\everymath={\displaystyle}
\renewcommand{\choicelabel}{\thechoice)}
\renewcommand{\choicelabel}{ 
  \ifnum\value{choice}>0
    \makebox[1.5cm][r]{\raggedleft \thechoice)}
  \else
   \raggedleft \thechoice)
  \fi
}

\setlength{\multicolsep}{0.6em}

\setlength{\columnseprule}{2pt}

\setlength{\columnsep}{2cm}
%%%%%% ---Comment out to add a header image ----

\begin{document}
%\begin{figure}[t]
%\includegraphics[width=1\textwidth,height=1.2\textheight,keepaspectratio]{header-cufm.png}
%\end{figure}

\begin{center}
\textbf{Números racionales y potencias} \\
Guía teórica\\
1er semestre 2024
\end{center}
\extraheadheight{-0.5in}

\runningheadrule \extraheadheight{0.15in}

\vspace{0.15in}
\runningheadrule \extraheadheight{0.14in}

\lhead{\ifcontinuation{Pregunta \ContinuedQuestion\ continua\ldots}{}}
\runningheader{Números racionales y potencias}{Guía teórica}{1er semestre 2024}
\runningfooter{}
              {\thepage\ de \numpages}
              {}
\vspace{0.05in}

\nopointsinmargin
\setlength\linefillthickness{0.1pt}
\setlength\answerlinelength{0.1in}
\vspace{0.1in}
%\parbox{6in}{
%\textbf{Objetivo:} Verificar el aprendizaje y comprensión de algunos de los temas fundamentales del nivel. }
%\vspace{0.15in}
\hrule 

\vspace{0.1in}
\parbox{6in}{
\textbf{Tema:} Operatoria con números enteros}
\vspace{0.15in}
\hrule

\begin{itemize}
  \item Adición en $\mathbb{Z}$: De iguales signos, se suman los valores absolutos y se conserva el signo. De distintos signos, se restan los valores absolutos y se conserva el signo del número de mayor valor absoluto.
  \item Sustracción en $\mathbb{Z}$: Se transforma en una adición y se procede como corresponde, considerando si los valores son de iguales o distintos signos.
  \item Multiplicación y división en $\mathbb{Z}$: De iguales signos, se multiplican o dividen los valores absolutos y el producto siempre es positivo. De distintos signos, se multiplican o dividen los valores absolutos y el producto siempre es negativo.
\end{itemize}

\parbox{6in}{
\textbf{Actividades propuestas}}
\vspace{0.15in}
\hrule 

\begin{questions}

\question Resuelve las adiciones y sustracciones.
\begin{multicols}{2}

\begin{itemize}
  \item[a.] $15+(-8)$
  \item[b.] $-41+(-32)$
  \item[c.] $-(-52)+25$
  \item[d.] $53+(-14)$
  \item[e.] $87-(-21)$
  \item[f.] $345+768-450$
  \item[g.] $45-(234-12)$
  \item[h.] $-(48-43+340)$
  \item[i.] $254-4.898+3.740$
  \item[j.] $-(-478-1.243+148)$
  \item[k.] $5+16+(-4)+25-45$
  \item[l.] $21-231-14+231+14$
\end{itemize}
\end{multicols}

\question Resuelve las multiplicaciones y divisiones.
\begin{multicols}{2}
  

  \begin{itemize}
    \item[a.] $12\cdot(-12)$
    \item[b.] $-(-42)\div 14$
    \item[c.] $72\div (-18)$
    \item[d.] $810\div (-90)$
    \item[e.] $12\cdot 76\div 3$
    \item[f.] $750\div(25\cdot 10)$
    \item[g.] $-(48\div 16\cdot 925)$
    \item[h.] $-(-7\cdot 1.199\cdot 2)$
    \item[i.] $35\div (-7)\cdot 4\div (-20)\cdot (-1)$
    \item[j.] $15\cdot 15\div (-25)\cdot 25\div 45$
    \item[k.] $289\div 17\cdot 17\cdot (-1)\div 17$
    \item[l.] $-66\div (-11)\cdot (-112)\div 12\cdot 10$
  \end{itemize}
\end{multicols}

  \question Resuelve las operaciones combinadas.

  \begin{multicols}{2}
  \begin{itemize}
  \item[a.]$2\cdot \left( 12\div (-6)+(-12)\cdot 6-48\div 16\right)$
  \item[b.]$-1-15+\left((-21)-21\cdot 3\right) +8\cdot 12$
  \item[c.]$\left( 19-120\div (-10)\cdot 6\cdot 10-10\right) \div 9$
  \item[d.]$7+\left( 200\div (-2)+(-9)\cdot 6\div 3+12\right)$
  \item[e.]$18-\left( 8+256\div 16\cdot 4-7\right) \cdot(-2)$
  \item[f.] $-100\div \left( 9\cdot 23\div 3- 45+1\right) \cdot 2$
  \item[g.] $8 + \left( 312\div 12\cdot 6\div  4-1\right) \cdot 2$
  \item[h.] $4\cdot \left( -2-205\div \left(65\div \left(-24+37\right)\right)\right) \cdot (-5)$
  \end{itemize}

\end{multicols}

  \question Resuelve los problemas

  \begin{itemize}
  \item[a. ] Una persona que está en el décimo quinto piso de un edificio, sube 5 pisos, luego baja 11 pisos y finalmente, sube otra cantidad incierta de pisos. Si desde esta última parada, desciendo al primer piso, pasando por 18 pisos, y sale del edificio, ¿cuántos pisos subió en la última parada?
  \item[b. ] Una deuda de \$18.000 se incrementó en \$4.000 cada 30 días, durante 180 dias. Luego de este periodo el monto total adeudado fue dividido en siete cuotas iguales y sin más intereses. ¿Cuánto dinero se debe pagar en cada cuota? ¿Qué número entero representa el monto total de la deuda?
  \end{itemize}
  \newpage

  \parbox{6in}{
    \textbf{Tema:} Operatoria con fracciones}
    \vspace{0.15in}
    \hrule
    
    \begin{itemize}
      \item Adición de fracciones: 
      $$\dfrac{a}{b}+\dfrac{c}{d}=\dfrac{a\cdot d+b\cdot c}{b\cdot d}$$
      \item Sustracción de fracciones: 
      $$\dfrac{a}{b}-\dfrac{c}{d}=\dfrac{a\cdot d-b\cdot c}{b\cdot d}$$
      \item Multiplicación de fracciones:
      $$\dfrac{a}{b}\cdot \dfrac{c}{d}=\dfrac{a\cdot c}{b\cdot d}$$
      \item División de fracciones:
      $$\dfrac{a}{b}\div \dfrac{c}{d}=\dfrac{a}{b}\cdot \dfrac{d}{c}=\dfrac{a\cdot d}{b\cdot c}$$
      \item Amplificar una fracción consiste en multiplicar su numerador y su denominador por un mismo valor.
      \item Simplificar una fracción consiste en dividir su numerador y su denominador por un mismo valor.
    \end{itemize}

    \parbox{6in}{
      \textbf{Actividades propuestas}}
      \vspace{0.15in}
      \hrule 

  \question Resuelve las operaciones entre fracciones. Expresa el resultado como fracción irreductible.
  \begin{multicols}{2}
    
  

  \begin{itemize}
    \item[a.] $\dfrac{3}{2}+\dfrac{5}{3}$
  \item[b.] $\dfrac{8}{9}-\dfrac{1}{2}$
  \item[c.] $\dfrac{12}{17}\cdot \dfrac{5}{6}$
  \item[d.] $\dfrac{8}{15}\div \dfrac{2}{5}$
  \item[e.] $\dfrac{7}{18}-\dfrac{4}{5}+\dfrac{5}{9}$
  \item[f.] $\dfrac{6}{7}\cdot\dfrac{7}{12}\div \dfrac{1}{2}$
  \item[g.] $\dfrac{2}{7}-\dfrac{1}{25}\div \dfrac{2}{3}$
  \item[h.] $\dfrac{21}{25}+\dfrac{24}{25}\cdot \dfrac{1}{12}$
  \item[i.] $\dfrac{4}{5}-\left(\dfrac{1}{5}+\dfrac{3}{8}\right)\div \dfrac{1}{2}$
  \item[j.] $\dfrac{12}{13}\cdot \left(\dfrac{5}{6}-\dfrac{2}{3}\right)-\dfrac{4}{26}$
  \item[k.] $\dfrac{5}{24}\cdot \left(\dfrac{3}{7}\div \dfrac{5}{4}\cdot \dfrac{4}{5}\right)+\dfrac{1}{2}$
  \item[l.] $\dfrac{13}{15}-\dfrac{13}{15}\cdot \dfrac{1}{2}\cdot \dfrac{2}{3}+\dfrac{13}{15}$
  \item[m.] $\left(\dfrac{9}{14}-\dfrac{3}{7}\right)\cdot \dfrac{7}{9}\div \left(\dfrac{1}{2}+\dfrac{1}{3}\right)$
  \item[n.] $\dfrac{21}{25}-\dfrac{7}{13}\cdot \left(\left(\dfrac{1}{2}\cdot \dfrac{2}{7}\right)+\dfrac{3}{5}\right)$
  \item[ñ.] $\dfrac{9}{34}\div \left(\dfrac{5}{23}\cdot \left(\dfrac{7}{2}-\dfrac{9}{10}\right)-\dfrac{14}{17}\right)$  
  \end{itemize}
\end{multicols}
  
\question Resulve los problemas.

\begin{itemize}
  \item[a.] La longitud de una pista de atletismo de cierto recinto deportivo es de 800 m. Si un atleta da 9 vueltas completas más tres cuartos de otra y, otro atleta completo 7 vueltas y un octavo de otra, ¿cuántos metros recorrieron entre ambos?
  \item[b.] Si un kilogramo de queso cuesta \$6.200, uno de paltas cuesta \$1.790 y uno de tomates, \$800, ¿cuánto dinero costará comprar un cuarto de kilogramo de queso, dos de paltas y uno de tomates?
\end{itemize}

\newpage

  \parbox{6in}{
    \textbf{Tema:} Operatoria con decimales}
    \vspace{0.15in}
    \hrule
    
    \begin{itemize}
      \item \textbf{Adición y sustracción de números decimales:} Se deben sumar o restar las cifras con igual orden posicional, de menor a mayor orden; es decir, las milésimas con las milésimas, las centésimas con las centésimas, décimas con décimas, unidades con unidades, decenas con decenas, etc. 
      
      \item \textbf{Multiplicación de números decimales:} Una técnica consiste en multiplicar los factores sin considerar la coma decimal, para luego ubicarla en el producto según el total de cifras decimales de ambos factores.
      \item \textbf{División de números decimales:} Una técnica consiste en amplificar, tanto el dividendo como el divisor, de tal manea que la división solo involucre npumeros naturales. Luego, resolver la división de forma tradicional.
    \end{itemize}

    \parbox{6in}{
      \textbf{Actividades propuestas}}
      \vspace{0.15in}
      \hrule 


      \question Resuelve las adiciones y sustracciones.
      \begin{multicols}{2}
        \begin{itemize}
          \item[a.] $15,2+8,9$
          \item[b.] $2,3-0,41$
          \item[c.] $24,78-5,673$
          \item[d.] $5,323+1,234$
          \item[e.] $11.817,34-1.021,324$
          \item[f.] $3,98+7,65-4,250$
          \item[g.] $4,25-(2,14-1,12)$
          \item[h.] $4,48-4,3+34,122$
          \item[i.] $2,254-0,828+0,172$
          \item[j.] $5-4,78-0,22+1,487$
          \item[k.] $0,13-0,110-0,001+3,2$
          \item[l.] $5,3+1,6-1,44-0,25$
          \item[m.] $2,1-1,31-0,77+0,02$
          \item[n.] $15-2,1-(11,2+0,5)$
          \item[ñ.] $7,42-6,881-(0,1+0,01)$       
        \end{itemize}
        
      \end{multicols}

      \question Resuelve las multiplicaciones y divisiones.
      \begin{multicols}{2}
        \begin{itemize}
          \item[a.] $0,21\cdot 0,21$
          \item[b.] $2,18\cdot 6$
          \item[c.] $4,2\div 0,4$
          \item[d.] $7,2\cdot 0,800$
          \item[e.] $85,10\div 9,2$
          \item[f.] $1,22\cdot 7,66\div 0,2$
          \item[g.] $1.086,5\div (2,65\cdot 100)$
          \item[h.] $4,881\div 0,03\cdot 0,95$
          \item[i.] $171,801\div (12,6\cdot 6,06)$
          \item[j.] $7,45\cdot 2,56\cdot 0,06$
          \item[k.] $3,5\div 7\cdot 4\div 0,2\cdot 1,45$
          \item[l.] $1,55\cdot 1,55\div 2,5\cdot 4,2\div 6$
          \item[m.] $37,6\div 1,6\cdot 1,3\cdot 12\div 0,4$
          \item[n.] $8,8\div 2,2\cdot 0,117\div 1,8\cdot 15$
          \item[ñ.] $2,88\cdot 3,3\div 8\cdot 12,4\div (3,6-2,8)$       
        \end{itemize}
        
      \end{multicols}

      \question Resuelve las operaciones combinadas.
      
        \begin{itemize}
          \item[a.] $4,2\cdot (1,12\div 0,4+3,42\cdot 6,5-4,2\div 6)$
          \item[b.] $1,98-1,75+(21,14-2,1\cdot 9)+0,8\cdot 1,1$
          \item[c.] $\left\{19,89-(120,5\div 0,5\cdot 0,1\cdot 0,8-0,28)\right\}\div 0,89$
          \item[d.] $38,84-(8,8+3,15\div 1,5\cdot 4)\cdot 2,2$
          \item[e.] $1,68+(9,5\cdot 2,1\div 3-4,2+1,1)\cdot 2,5$
          \item[f.] $0,8+(6\div 0,75\cdot 8\div 4-1,7)\cdot 7,1$
                
        \end{itemize}
        
      
\newpage
      \question Resuelve los problemas.

\begin{itemize}
  \item[a.] ¿Cuántos vasos de $0,25$ L es posible llenar con un bidón de 5 L?
  \item[b.] Si la capacidad de almacenajo de un pendrive es de $1,9$ GB, ¿cuánto espacio queda disponible al guardar en él un software de 512 MB? Para resolver considera que 1 GB $=$ 1.024 MB.
  \item[c.] Encuentra un patron en la secuencia numérica y determina el valor de falta en ella:
  $$2\rightarrow 3,6\rightarrow 3,24\rightarrow \rule{1cm}{0.4pt}\rightarrow 0,32805$$ 
\end{itemize}  



  \parbox{6in}{
    \textbf{Tema:} Transformcación y operatoria con números decimales infinitos}
    \vspace{0.15in}
    \hrule
    
    \begin{itemize}
      \item \textbf{Números decimales infinitos periódicos:} Su parte decimal está compuesta por una o más cifras que se repiten infinitas veces. Por ejemplo: $2,\overline{3}=2,3333\ldots$ y $0,\overline{12}=0,121212\ldots$
      
      \item \textbf{Números decimales infinitos semiperiódicos:} Su parte decimal está compuesta por una o más cifras, seguidas de una o más cifras que se repiten infinitas veces. Por ejemplo: $1,2\overline{4}=1,24444\ldots$ y $7,21\overline{430}=7,21430430430\ldots$
      \item \textbf{De número decimal finito a fracción:} Se escribe todo el número decimal como el numerador, sin la coma, y en el denominador se escribe un 1 y tantos 0 como cifras decimales tenga el número.
      \item \textbf{De fracción a decimal:} Para transformar una fracción a número decimal se resuelve la división del numerador por el denominador.
      \item Si el número tiene periodo (sea periódico o anteperiódico), en el numerador se escribe la diferencia entre el \textbf{número completo} y la parte \textbf{no periódica}, y en el denominador tantos 9 como cifras tenga el \textbf{periodo} segidos de tantos 0 como digitos tenga el \textbf{anteperiodo}.
    \end{itemize}

    \parbox{6in}{
      \textbf{Actividades propuestas}}
      \vspace{0.15in}
      \hrule 


      \question Representa como número decimal finito, infinito periódico o infinito semiperiódico.
      \begin{multicols}{4}
        \begin{itemize}
          \item[a.] $\dfrac{3}{20}$
          \item[b.] $\dfrac{7}{3}$
          \item[c.] $\dfrac{5}{45}$
          \item[d.] $\dfrac{1}{5}$
          \item[e.] $\dfrac{181}{90}$
          \item[f.] $\dfrac{13}{99}$
          \item[g.] $\dfrac{11}{90}$
          \item[h.] $\dfrac{50}{90}$
          \item[i.] $\dfrac{3}{4}$
          \item[j.] $\dfrac{54}{990}$
          \item[k.] $\dfrac{7}{70}$
          \item[l.] $\dfrac{12}{512}$
          \item[m.] $\dfrac{100}{45}$
          \item[n.] $\dfrac{7}{330}$
          \item[ñ.] $\dfrac{1}{20}$
          
        \end{itemize}
        
      \end{multicols}
      \question Representa los números decimales como fracción.
      \begin{multicols}{4}
        \begin{itemize}
          \item[a.] $12,87$
          \item[b.] $23,5$
          \item[c.] $21,\overline{57}$
          \item[d.] $61,0\overline{5}$
          \item[e.] $111,1\overline{1}$
          \item[f.] $0,\overline{501}$
          \item[g.] $34,22\overline{27}$
          \item[h.] $1.600,00\overline{90}$
          \item[i.] $78,112\overline{112}$
          \item[j.] $15,\overline{90}$
          \item[k.] $0,\overline{215}$
          \item[l.] $23,\overline{5}$
          \item[m.] $0,1\overline{50}$
          \item[n.] $91,\overline{58}$
          \item[ñ.] $0,2\overline{971}$
          
        \end{itemize}
        
      \end{multicols}

    \question Resuelve las operaciones combinadas.
    \begin{multicols}{2}
      \begin{itemize}
        \item[a.] $\left(0,\overline{71}+\dfrac{29}{99}-0,\overline{01}\right)\cdot 4,8\overline{48}\cdot 99$
        \item[b.] $9,5\div \left(0,\overline{7}\cdot 9+\dfrac{37}{10}\cdot 0,0\overline{01}\right)\cdot 693,37$
      \end{itemize}
      
    \end{multicols}


\end{questions}
\end{document}