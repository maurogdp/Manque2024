\documentclass[spanish,letterpaper, 11pt, addpoints, answers]{exam}
\usepackage[left=2cm, right=2cm, top=2cm, bottom=2.5cm]{geometry}
\PassOptionsToPackage{T1}{fontenc} 
    \usepackage{fontenc} 
    \usepackage[utf8]{inputenc}
    % Cargar babel y configurar para español
    \usepackage[spanish, es-tabla, es-noshorthands]{babel}
    \usepackage{lmodern}
    \usepackage{amsfonts}
    \usepackage{multirow}
    \usepackage{hhline}
    \usepackage[none]{hyphenat}
\usepackage[utf8]{inputenc}
\usepackage{graphics}
\usepackage{color}
\usepackage{amssymb}
\usepackage{amsmath}
\usepackage{enumitem}
\usepackage{xcolor}
\usepackage{cancel}
\usepackage{ragged2e}
\usepackage{graphicx}
\usepackage{multicol}
\usepackage{color}
\usepackage{tikz}
\pointpoints{Punto}{Puntos}

\usepackage{graphicx}
\usepackage{tikz}
\usetikzlibrary{babel,arrows.meta,decorations.pathmorphing, backgrounds,positioning,fit,petri, shapes, shadows}

\usepackage{tikz,color}
\usepackage{pgf-pie} 

\usepackage{stackengine}
\newcommand\xrowht[2][0]{\addstackgap[.5\dimexpr#2\relax]{\vphantom{#1}}}



\CorrectChoiceEmphasis{\color{red}}
\noprintanswers

\renewcommand{\choiceshook}{%
    \setlength{\leftmargin}{30pt}%
}


\everymath={\displaystyle}
\renewcommand{\choicelabel}{\thechoice)}
\renewcommand{\choicelabel}{ 
  \ifnum\value{choice}>0
    \makebox[1.5cm][r]{\raggedleft \thechoice)}
  \else
   \raggedleft \thechoice)
  \fi
}

\setlength{\multicolsep}{0.6em}

\setlength{\columnseprule}{2pt}

\setlength{\columnsep}{2cm}
%%%%%% ---Comment out to add a header image ----

\begin{document}
%\begin{figure}[t]
%\includegraphics[width=1\textwidth,height=1.2\textheight,keepaspectratio]{header-cufm.png}
%\end{figure}

\begin{center}
  \textbf{Números racionales y potencias} \\
  Guía teórica\\
  1er semestre 2024
\end{center}
\extraheadheight{-0.5in}

\runningheadrule \extraheadheight{0.15in}

\vspace{0.15in}
\runningheadrule \extraheadheight{0.14in}

\lhead{\ifcontinuation{Pregunta \ContinuedQuestion\ continua\ldots}{}}
\runningheader{Números racionales y potencias}{Guía teórica}{1er semestre 2024}
\runningfooter{}
{\thepage\ de \numpages}
{}
\vspace{0.05in}

\nopointsinmargin
\setlength\linefillthickness{0.1pt}
\setlength\answerlinelength{0.1in}
\vspace{0.1in}
%\parbox{6in}{
%\textbf{Objetivo:} Verificar el aprendizaje y comprensión de algunos de los temas fundamentales del nivel. }
%\vspace{0.15in}
\hrule

\vspace{0.1in}
\parbox{6in}{
  \textbf{Tema:} Números racionales}
\vspace{0.15in}
\hrule

\begin{itemize}
  \item El conjunto de los números racionales $\mathbb{Q}$: El conjunto de los números racionales se define como el conjunto de totas aquellas fracciones entre números enteros donde el denominador sea distinto de 0. De manera más formal, $\mathbb{Q}=\left\{\dfrac{a}{b} / a,b \in\mathbb{Z} \wedge b\neq 0\right\}$.
  \item Orden en $\mathbb{Q}$: Una técnica para saber que número racional es menor o mayor que otro consiste en comparar las fraccioness de la siguiente manera.
        \begin{center}
          $\dfrac{a}{b}<\dfrac{c}{d}\Leftrightarrow a\cdot d < b\cdot c$,\hspace{0.5cm}o bien,\hspace{0.5cm}    $\dfrac{a}{b}>\dfrac{c}{d}\Leftrightarrow a\cdot d > b\cdot c$
        \end{center}
  \item Subconjuntos de $\mathbb{Q}$: Todo número naturak o entero es posible escribirlo como un fracción cuyo numersdor es dicho valor y su denominador es 1. Luego: $\mathbb{N}\subset \mathbb{Q}$ y también $\mathbb{N}\subset \mathbb{Q}$.
  \item Transformación de decimales periódicos y semiperiódicos negativos a fracción: Para estos casos, se procede como de costumbre snenponiendo el signo ``$-$'' al resultado.
        \begin{center}
          $-3{,}\overline{56}=-\dfrac{356-3}{99}=-\dfrac{353}{99}$\hspace{2cm}$-34{,}96\overline{7}=-\dfrac{34967-3496}{900}=-\dfrac{31471}{900}$
        \end{center}
\end{itemize}

\parbox{6in}{
  \textbf{Actividades propuestas}}
\vspace{0.15in}
\hrule

\begin{questions}

  \question Identifica si cada número dado pertenece a $\mathbb{N}$, $\mathbb{Z}$ y/o $\mathbb{Q}$.
  \begin{multicols}{4}

    \begin{itemize}
      \item[a.] $-8$
      \item[b.] $432$
      \item[c.] $-52{,}\overline{25}$
      \item[d.] $5,\overline{78}$
      \item[e.] $-35,5\overline{43}$
      \item[f.] $\dfrac{3}{6}$
      \item[g.] $23{,}0\overline{01}$
      \item[h.] $-4\dfrac{2}{3}$
      \item[i.] $-\dfrac{43}{990}$
    \end{itemize}
  \end{multicols}

  \question Ordena de menor a mayor los siguientes conjuntos de números racionales.
  \begin{multicols}{2}


    \begin{itemize}
      \item[a.] $5{,}1;5{,}\overline{01};5{,}0\overline{1};5{,}\overline{1};5{,}0\overline{01};5{,}\overline{001};5{,}00\overline{1}$
      \item[b.] $-0{,}1;-0{,}\overline{01};-0{,}0\overline{1};-0{,}\overline{1};-0{,}\overline{001};-0{,}00\overline{1}$
      \item[c.] $\dfrac{3}{5};\dfrac{5}{3};\dfrac{1}{9};\dfrac{1}{3};\dfrac{5}{9};\dfrac{35}{99};\dfrac{53}{99};\dfrac{11}{90}$
      \item[d.] $-\dfrac{7}{9};-\dfrac{9}{7};-\dfrac{9}{90};-\dfrac{79}{99};-\dfrac{97}{99};-\dfrac{7}{90};-\dfrac{79}{90};-\dfrac{97}{90}$
      \item[e.] $-3{,}\overline{5};-\dfrac{7}{2};-14{,}0\overline{6};0{,}\overline{25};0{,}0\overline{25};-\dfrac{50}{3};\dfrac{1}{4}$
      \item[f.] $12{,}15;\dfrac{5}{90};-3{,}6\overline{64};10{,}\overline{051};10{,}0\overline{51};0{,}0\overline{05};\dfrac{1}{2}$
    \end{itemize}
  \end{multicols}

  \question Escribe $\mathbb{Q}^+$ o $\mathbb{Q}^-$ si la variable es posible representarla por un npumero racional positivo o uno negativo. De ser posible representarla con ambos, escribe $\mathbb{Q}$.

  \begin{multicols}{2}
    \begin{itemize}
      \item[a.] Tiempo en horas.
      \item[b.] Estatura en metros.
      \item[c.] Temperatura en grados Celsius.
      \item[d.] Saldo en la cuenta corriente.
      \item[e.] Profundidad de un submarino.
      \item[f.] Masa corporal en kilogramos.
    \end{itemize}

  \end{multicols}

  \newpage

  \parbox{6in}{
    \textbf{Tema:} Aproximación de números decimales}
  \vspace{0.15in}
  \hrule

  \begin{itemize}
    \item Redondeo: Redondear un número a una determinada cifra es una técnica de aproximación que considera la cifra decimal inmediatamente siguiente a la que determine la aproximación, según el siguiente criterio:
          \begin{itemize}
            \item Si dicha cifra es menor que 5,las cifras se conservan.
            \item Si dicha cifra es igual o mayor que 5, la cifra por aproximar aumenta una unidad.
          \end{itemize}
    \item Truncamiento: Se consideran solo las cifras decimales a partir de un orden posicional determinado.
    \item Cuando se aproxime un número, ya sea por redondeo o truncamiento, se dirá que ha sido pot defecto si dicha aproximación es menor al valor exacto; mientras que si es mayor, se dira que es por exceso.
    \item El error absoluto corresponde a la diferencia, en valor absoluto, entre el valor a aproximar y la aproximación. Este valor se da al aproximar con cualquier técnica.
  \end{itemize}

  \parbox{6in}{
    \textbf{Actividades propuestas}}
  \vspace{0.15in}
  \hrule

  \question Redondea a la décima, centésima y luego a la milésima los siguientes números decimales.
  \begin{multicols}{3}



    \begin{itemize}
      \item[a.] $-34{,}25$
      \item[b.] $1.328{,}05$
      \item[c.] $-445{,}44$
      \item[d.] $2.453{,}0\overline{91}$
      \item[e.] $15{,}\overline{50}$
      \item[f.] $-63{,}0\overline{53}$
      \item[g.] $79{,}81$
      \item[h.] $378{,}99$
      \item[i.] $-863{,}0\overline{9}$
      \item[j.] $0{,}0007$
      \item[k.] $990{,}8\overline{5}$
      \item[l.] $1.256{,}\overline{99}$
    \end{itemize}
  \end{multicols}

  \question Trunca a la décima, centésima y luego a la milésima los siguientes números decimales

  \begin{multicols}{3}



    \begin{itemize}
      \item[a.] $-34{,}25$
      \item[b.] $1.328{,}05$
      \item[c.] $-445{,}44$
      \item[d.] $2.453{,}0\overline{91}$
      \item[e.] $15{,}\overline{50}$
      \item[f.] $-63{,}0\overline{53}$
      \item[g.] $79{,}81$
      \item[h.] $378{,}99$
      \item[i.] $-863{,}0\overline{9}$
      \item[j.] $0{,}0007$
      \item[k.] $990{,}8\overline{5}$
      \item[l.] $1.256{,}\overline{99}$
    \end{itemize}
  \end{multicols}

  \question Señala si las aproximaciones de las actividades anteriores fueron por defecto o por exceso.

  \question Calcula el error absoluto de cada aproximación realizada en las actividades anteriores.

  \newpage

  \parbox{6in}{
    \textbf{Tema:} Adición y sustracción en $\mathbb{Q}$, Propiedades.}
  \vspace{0.15in}
  \hrule

  \begin{itemize}
    \item \textbf{Adición y sustracción en $\mathbb{Q}$:} La adición y sustracción de números racionales son operaciones binarias que pueden ser resueltas aplicando:
          \begin{center}
            $\dfrac{a}{b}+\dfrac{c}{d}=\dfrac{a\cdot d+b\cdot c}{b\cdot d}$\hspace{1cm} $\dfrac{a}{b}-\dfrac{c}{d}=\dfrac{a\cdot d-b\cdot c}{b\cdot d}$
          \end{center}

    \item \textbf{Clausura:} Al sumar números racionales siempre se obtiene un número racional.
    \item \textbf{Asociativa:} Sean $a$, $b$ y $c$ tres números racionles: $a+(b+c)=(a+b)+c$
    \item \textbf{Conmutativa:} Sean $a$ y $b$ dos números racionales: $a+b=b+a$
    \item \textbf{Elemento neutro:} Sea $a$ un número racional: $a+0=a$
    \item \textbf{Inverso aditivo (Opuesto):} Sea $a$ un número racional: $a+(-a)=0$. El inverso aditivo de $a$ es $-a$
    \item Restar un número racional equivale a sumar su inverso aditivo, es decir, $\dfrac{a}{b}-\dfrac{c}{d}=\dfrac{a}{b}+\left(-\dfrac{c}{d}\right)$.
  \end{itemize}

  \parbox{6in}{
    \textbf{Actividades propuestas}}
  \vspace{0.15in}
  \hrule


  \question Resuelve las adiciones de números racionales. Expresa el resultado como fracción irreductible.
  \begin{multicols}{2}
    \begin{itemize}
      \item[a.] $-\dfrac{1}{5}+\dfrac{1}{2}$
      \item[b.] $-\dfrac{7}{8}+\dfrac{1}{4}$
      \item[c.] $-\dfrac{19}{17}+\dfrac{1}{5}$
      \item[d.] $\dfrac{25}{4}+\left(-\dfrac{1}{8}\right)$
      \item[e.] $-\dfrac{1}{4}+\dfrac{7}{8}+\dfrac{15}{2}$
      \item[f.] $-\dfrac{5}{6}+\dfrac{7}{9}+\dfrac{1}{3}$
      \item[g.] $\dfrac{25}{12}+\left(-\dfrac{5}{6}\right)+\left(-\dfrac{15}{4}\right)$
      \item[h.] $-\dfrac{1}{2}+\dfrac{17}{24}+\dfrac{7}{8}$
      \item[i.] $\dfrac{3}{6}+0{,}45+\left(-\dfrac{1}{7}\right)$
      \item[j.] $-\dfrac{1}{2}+0{,}9+\left(-\dfrac{3}{10}\right)$
      \item[k.] $\dfrac{7}{9}+0{,}\overline{5}+\left(-\dfrac{24}{18}\right)$
      \item[l.] $45{,}\overline{45}+\left(-\dfrac{500}{11}\right)+\dfrac{11}{500}$
    \end{itemize}

  \end{multicols}

  \question Resuelve las sustracciones de números racionales. Expresa el resultado como fracción irrecductible.
  \begin{multicols}{2}
    \begin{itemize}
      \item[a.] $-\dfrac{3}{7}-\dfrac{2}{5}$
      \item[b.] $-\dfrac{6}{11}-\dfrac{5}{12}$
      \item[c.] $-\dfrac{21}{19}-\dfrac{2}{3}$
      \item[d.] $\dfrac{12}{7}-\left(-\dfrac{5}{2}\right)$
      \item[e.] $-\dfrac{21}{4}-\dfrac{5}{6}-\dfrac{7}{12}$
      \item[f.] $-\dfrac{3}{4}-\dfrac{3}{8}-\dfrac{7}{12}$
      \item[g.] $\dfrac{28}{25}-\left(-\dfrac{3}{5}\right)-\left(-\dfrac{1}{10}\right)$
      \item[h.] $-\dfrac{2}{75}-\dfrac{7}{25}-\dfrac{3}{50}$
      \item[i.] $\dfrac{1}{2}-0{,}33-\left(-\dfrac{4}{25}\right)$
      \item[j.] $-\dfrac{4}{7}-1{,}6-\left(-\dfrac{3}{35}\right)$
      \item[k.] $\dfrac{4}{9}-0{,}3\overline{1}-\left(-\dfrac{2}{5}\right)$
      \item[l.] $87{,}\overline{78}-\left(-\dfrac{103}{33}\right)-\dfrac{1}{11}$
    \end{itemize}

  \end{multicols}

  \question Resuelve las adiciones y sustracciones de números racionales. Expresa el resultado como fracción irreductible.
  \begin{multicols}{2}
  \begin{itemize}
    \item[a.] $-\dfrac{4}{5}+\dfrac{8}{7}+\dfrac{2}{35}$
    \item[b.] $-\dfrac{5}{2}-\dfrac{8}{3}+\dfrac{5}{6}$
    \item[c.] $\dfrac{2}{3}+\dfrac{3}{4}-\dfrac{17}{24}$
    \item[d.] $\dfrac{4}{9}-\dfrac{1}{3}+\dfrac{5}{2}$
    \item[e.] $\dfrac{1}{2}-\dfrac{7}{8}-\dfrac{3}{4}$
    \item[f.] $-\dfrac{5}{12}+\dfrac{5}{3}-\dfrac{5}{6}$
    \item[g.] $-\dfrac{7}{2}+\dfrac{4}{5}-\dfrac{3}{4}$
    \item[h.] $\dfrac{15}{9}-\dfrac{17}{36}+\dfrac{23}{18}$
    \item[i.] $\dfrac{3}{4}+\dfrac{4}{5}-\dfrac{12}{25}$
    \item[j.] $-\dfrac{1}{60}+\dfrac{1}{40}+\dfrac{1}{100}$
    \item[k.] $-\dfrac{13}{25}-(-0{,}2)-0{,}05$
    \item[l.] $-\dfrac{4}{9}+0{,}\overline{01}-\left(-\dfrac{9}{11}\right)$
    \item[m.] $-\dfrac{2}{3}-0{,}2\overline{3}-\left(-5{,}\overline{3}\right)$
    \item[n.] $-\left(-\dfrac{51}{18}\right)-0{,}\overline{4}+\dfrac{5}{6}$
    \item[ñ.] $-\left(-\dfrac{5}{9}\right)-\left(-0{,}3\right)-\dfrac{7}{90}$

  \end{itemize}
\end{multicols}

  \question Expresa los resultados de la actividad 10 como números decimales.

  \question Utiliza la siguiente tabla y resuelve las siguientes operaciones dadas por los valores correspondientes a cada par de números. Expresa el resultado como fracción irreductible.

  \begin{center}
    \begin{tabular}{|c|c|c|c|c|c|c|}\hline
      \textbf{Puntos}&\textbf{1}&\textbf{2}&\textbf{3}&\textbf{4}&\textbf{5}&\textbf{6}\\ \hline \xrowht{25pt} 
      \textbf{1}&$0{,}8$&$\dfrac{3}{5}$&$-10{,}\overline{6}$&$-0{,}\overline{01}$&$-3{,}21$&$41{,}1\overline{2}$\\ \hline \xrowht{25pt}
      \textbf{2}&$0{,}1\overline{7}$&$-2{,}\overline{21}$&$-0{,}0\overline{01}$&$1{,}1\overline{21}$&$-\dfrac{1}{90}$&$0{,}\overline{7}$\\ \hline \xrowht{25pt} 
      \textbf{3}&$0{,}\overline{8}$&$0{,}5$&$\dfrac{8}{99}$&$-2{,}0\overline{2}$&$-3{,}\overline{03}$&$5{,}\overline{01}$\\ \hline \xrowht{25pt} 
      \textbf{4}&$-\dfrac{1}{6}$&$7{,}0\overline{70}$&$-1{,}1\overline{8}$&$-2{,}1$&$-4{,}\overline{5}$&$0{,}6$\\ \hline \xrowht{25pt} 
      \textbf{5}&$5{,}\overline{14}$&$3{,}\overline{21}$&$-\dfrac{4}{9}$&$1{,}\overline{1}$&$21{,}9\overline{89}$&$41{,}0\overline{1}$\\ \hline \xrowht{25pt} 
      \textbf{6}&$-4{,}8$&$0{,}4\overline{5}$&$-12{,}\overline{3}$&$0{,}\overline{32}$&$3{,}3\overline{29}$&$\dfrac{7}{2}$\\ \hline

      
    \end{tabular}
  \end{center}

  \begin{multicols}{2}
    \begin{itemize}
      \item[a.] (2 y 2) $+$ (1 y 5)
      \item[b.] (1 y 6) $+$ (3 y 6)
      \item[c.] (3 y 4) $+$ (1 y 4)
      \item[d.] (4 y 3) $+$ (2 y 5)
      \item[e.] (6 y 2) $+$ (3 y 3)
      \item[f.] (5 y 3) $-$ (1 y 1)
      \item[g.] (4 y 1) $-$ (6 y 6)
      \item[h.] (4 y 6) $-$ (4 y 3)
      \item[i.] (6 y 1) $-$ (2 y 5)
      \item[j.] (3 y 3) $-$ (4 y 4)
      \item[k.] (5 y 1) $+$ (1 y 1) $-$ (3 y 5)
      \item[l.] (6 y 5) $+$ (2 y 4) $-$ (4 y 2)
      \item[m.] (6 y 6) $+$ (1 y 6) $+$ (3 y 1)
      \item[n.] (4 y 4) $-$ (2 y 5) $+$ (4 y 5)
      \item[ñ.] (1 y 5) $+$ (4 y 3) $-$ (5 y 6)
    \end{itemize}

  \end{multicols}

  \question Expresa los resultados de la actividad 12 como números decimales.

  \newpage

  \parbox{6in}{
    \textbf{Tema:} Multiplicación y división en $\mathbb{Q}$, Propiedades.}
  \vspace{0.15in}
  \hrule

  \begin{itemize}
    \item \textbf{Multiplicación y división en $\mathbb{Q}$:} La multiplicación y división de números racionales son operaciones binarias que pueden ser resueltas aplicando:
          \begin{center}
            $\dfrac{a}{b}\cdot\dfrac{c}{d}=\dfrac{a\cdot c}{b\cdot d}$\hspace{1cm} $\dfrac{a}{b}\div\dfrac{c}{d}=\dfrac{a}{b}\cdot \dfrac{d}{c}=\dfrac{a\cdot d}{b\cdot c}$
          \end{center}

    \item \textbf{Clausura:} Al multiplicar números racionales siempre se obtiene un número racional.
    \item \textbf{Asociativa:} Sean $a$, $b$ y $c$ tres números racionles: $a\cdot(b\cdot c)=(a\cdot b)\cdot c$
    \item \textbf{Conmutativa:} Sean $a$ y $b$ dos números racionales: $a\cdot b=b\cdot a$
    \item \textbf{Elemento neutro:} Sea $a$ un número racional: $a\cdot 1=a$
    \item \textbf{Inverso multiplicativo (Recíproco):} Sea $a$ un número racional: $a\cdot \dfrac{1}{a}=1$. El inverso multiplicativo de $a$ es $\dfrac{1}{a}$
    \item Dividir números racionales equivale a multiplicar por el inverso multiplicativo del divisor, es decir, $\dfrac{a}{b}\div \dfrac{c}{d}=\dfrac{a}{b}\cdot \left(\dfrac{d}{c}\right)$.
  \end{itemize}

  \parbox{6in}{
    \textbf{Actividades propuestas}}
  \vspace{0.15in}
  \hrule


  \question Resuelve las multiplicaciones de números racionales. Expresa el resultado como fracción irreductible.

  \begin{multicols}{2}
    \begin{itemize}
      \item[a.] $-\dfrac{2}{7}\cdot \dfrac{3}{8}$
      \item[b.] $-\dfrac{5}{12}\cdot \dfrac{6}{25}$
      \item[c.] $-\dfrac{18}{13}\cdot \dfrac{13}{54}$
      \item[d.] $\dfrac{35}{7}\cdot \left(-\dfrac{9}{5}\right)$
      \item[e.] $-\dfrac{1}{4}\cdot \dfrac{8}{7}\cdot \dfrac{2}{17}$
      \item[f.] $-\dfrac{6}{11}\cdot \dfrac{7}{9}\cdot \dfrac{1}{2}$
      \item[g.] $\dfrac{24}{13}\cdot \left(-\dfrac{39}{6}\right)\cdot \left(-\dfrac{13}{4}\right)$
      \item[h.] $-\dfrac{7}{8}\cdot \dfrac{12}{49}\cdot \dfrac{1}{2}$
      \item[i.] $\dfrac{10}{29}\cdot 1{,}33\cdot \left(-\dfrac{58}{3}\right)$
      \item[j.] $-\dfrac{4}{5}\cdot 0{,}8\cdot \left(-\dfrac{5}{4}\right)$
      \item[k.] $\dfrac{9}{8}\cdot 0{,}3\overline{5}\cdot \left(-\dfrac{25}{12}\right)$
      \item[l.] $23{,}\overline{23}\cdot \left(-\dfrac{11}{100}\right)\cdot \dfrac{9}{46}$
    \end{itemize}
  \end{multicols}

  \question Resuelve las divisiones de números racionales. Expresa el resultado como fracción irreductible.

  \begin{multicols}{2}
    \begin{itemize}
      \item[a.] $-\dfrac{3}{16}\div \dfrac{5}{8}$
      \item[b.] $-\dfrac{6}{11}\div \dfrac{15}{22}$
      \item[c.] $-\dfrac{24}{9}\div \dfrac{8}{3}$
      \item[d.] $\dfrac{150}{49}\div \left(-\dfrac{50}{7}\right)$
      \item[e.] $-\dfrac{20}{5}\div \dfrac{16}{25}\div \dfrac{1}{2}$
      \item[f.] $-\dfrac{5}{7}\div \dfrac{4}{3}\div \dfrac{2}{9}$
      \item[g.] $\dfrac{21}{17}\div \left(-\dfrac{7}{34}\right)\div \left(-\dfrac{1}{10}\right)$
      \item[h.] $-d\frac{4}{7}\div \dfrac{7}{4}\div \dfrac{16}{49}$
      \item[i.] $\dfrac{1}{8}\div 0{,}25\div \left(-\dfrac{8}{31}\right)$
      \item[j.] $-\dfrac{7}{9}\div 1{,}5\div \left(-\dfrac{259}{99}\right)$
      \item[k.] $\dfrac{1}{2}\div 0{,}\overline{3}\div \left(-0{,}25\right)$
      \item[l.] $1{,}\overline{798}\div \left(-\dfrac{11}{2}\right)\div \dfrac{599}{9}$
    \end{itemize}
  \end{multicols}

  \question Resuelve las operaciones de números racionales. Expresa el resultado como fracción irreductible.
  \begin{multicols}{2}


    \begin{itemize}
      \item[a.] $\dfrac{2}{7}\cdot \dfrac{1}{2}-\dfrac{9}{14}$
      \item[b.] $-\dfrac{5}{2}+\dfrac{5}{4}\cdot \dfrac{1}{8}$
      \item[c.] $\dfrac{3}{14}+\dfrac{3}{4}\div \dfrac{7}{4}$
      \item[d.] $\dfrac{2}{9}\div \dfrac{4}{7}-\dfrac{3}{4}$
      \item[e.] $\dfrac{3}{100}+\dfrac{1}{5}\div \dfrac{5}{12}$
      \item[f.] $-\dfrac{1}{6}+\dfrac{5}{4}\div \dfrac{7}{11}$
      \item[g.] $-\dfrac{5}{2}-\dfrac{9}{11}\div \dfrac{7}{12}$
      \item[h.] $\dfrac{3}{10}\div \dfrac{9}{16}+\dfrac{8}{18}$
      \item[i.] $\dfrac{1}{4}-\dfrac{2}{3}\div \dfrac{9}{25}$
      \item[j.] $-\dfrac{5}{60}\cdot \dfrac{3}{2}-\dfrac{5}{90}$
      \item[k.] $-\dfrac{24}{13}+(-0{,}2)\div 0{,}05$
      \item[l.] $-\dfrac{99}{990}\cdot 0{,}\overline{01}+\left(-\dfrac{9}{10}\right)$
      \item[m.] $-\dfrac{3}{5}-0{,}0\overline{3}\div \left(-4{,}\overline{3}\right)$
      \item[n.] $-\left(-\dfrac{21}{12}\right)\div 0{,}\overline{7}-\dfrac{3}{8}$
    \end{itemize}
  \end{multicols}

  \question Transforma los siguientes enunciados del lenguaje natural en expresiones numéricas y calcula su resultado.

  \begin{itemize}
    \item[a.] Resta el cuadrado del número 5 al doble de la adición de 3 y 9.
    \item[b.] Divide el cuadrado de la diferencia entre 17 y 5 por el triple de la adición de 5 y 3.
    \item[c.] Eleva a tres la adición entre $0{,}7$ y $2{,}3$ y disminuye su resultado por el cuádruple de la diferencia entre $8{,}7$ y $5{,}2$.
    \item[d.] El producto entre el número 8 y la adición de sus primeros dos sucesores se aumenta por el triple de la diferencia entre $115{,}7$ y $7{,}7$.
  \end{itemize}

  \question Transforma la expresión numérica a lenguaje natural y calcula el resultado.
  \begin{multicols}{2}
    \begin{itemize}
      \item[a.] $3(3-9)+10$
      \item[b.] $6(7+8)+2(32{,}7-12{,}34)$
      \item[c.] $(4-17)^2$
      \item[d.] $(3+8)^2\div 11$
    \end{itemize}

  \end{multicols}

\end{questions}
\end{document}