\documentclass[spanish,letterpaper, 12pt, addpoints, answers]{exam}
\usepackage[left=2cm, right=2cm, top=2cm, bottom=2.5cm]{geometry}
\PassOptionsToPackage{T1}{fontenc} 
    \usepackage{fontenc} 
    \usepackage[utf8]{inputenc}
    % Cargar babel y configurar para español
    \usepackage[spanish, es-tabla, es-noshorthands]{babel}
    \usepackage{lmodern}
    \usepackage{amsfonts}
    \usepackage{multirow}
    \usepackage{hhline}
    \usepackage[none]{hyphenat}
\usepackage[utf8]{inputenc}
\usepackage{graphics}
\usepackage{color}
\usepackage{amssymb}
\usepackage{amsmath}
\usepackage{enumitem}
\usepackage{xcolor}
\usepackage{cancel}
\usepackage{ragged2e}
\usepackage{graphicx}
\usepackage{multicol}
\usepackage{color}
\usepackage{tikz}
\pointpoints{Punto}{Puntos}

\usepackage{graphicx}
\usepackage{tikz}
\usetikzlibrary{babel,arrows.meta,decorations.pathmorphing, backgrounds,positioning,fit,petri, shapes, shadows}

\usepackage{tikz,color}
\usepackage{pgf-pie} 



\CorrectChoiceEmphasis{\color{red}}
%\noprintanswers

\renewcommand{\choiceshook}{%
    \setlength{\leftmargin}{30pt}%
}


\everymath={\displaystyle}
\renewcommand{\choicelabel}{\thechoice)}
\renewcommand{\choicelabel}{ 
  \ifnum\value{choice}>0
    \makebox[1.5cm][r]{\raggedleft \thechoice)}
  \else
   \raggedleft \thechoice)
  \fi
}

\setlength{\multicolsep}{0.6em}

\setlength{\columnseprule}{2pt}

\setlength{\columnsep}{2cm}
%%%%%% ---Comment out to add a header image ----

\author{Mauro Díaz Poblete}
\begin{document}
%\begin{figure}[t]
%\includegraphics[width=1\textwidth,height=1.2\textheight,keepaspectratio]{header-cufm.png}
%\end{figure}

\begin{center}
    \textbf{Evaluación Números - Primero medio}
\end{center}
\extraheadheight{-0.5in}

\textbf{Nombre:}\rule{9cm}{0.5pt}\hspace{1cm}\textbf{Curso:}\rule{2cm}{0.5pt}
\vspace{0.5cm}

\textbf{Fecha:}\rule{4cm}{0.5pt}\hspace{1cm}\textbf{Puntaje obtenido:}\rule{2cm}{0.5pt}\textbf{Nota:}\rule{2cm}{0.5pt}

\runningheadrule \extraheadheight{0.15in}

\vspace{0.05in}
\runningheadrule \extraheadheight{0.14in}

\lhead{\ifcontinuation{Pregunta \ContinuedQuestion\ continua\ldots}{}}
\runningheader{Matemática}{Evaluación Números}{Primerio medio}
\runningfooter{}
{\thepage\ de \numpages}
{}
\vspace{0.05in}

\nopointsinmargin
\setlength\linefillthickness{0.1pt}
\setlength\answerlinelength{0.1in}
\vspace{0.05in}
%\parbox{6in}{
%\textbf{Objetivo:} Verificar el aprendizaje y comprensión de algunos de los temas fundamentales del nivel. }
\vspace{0.1in}

\begin{center}
    \fbox{\fbox{\parbox{16cm}{
                {\textbf{Objetivo General: }OA1\\

                        \textbf{Objetivos específicos:}
                        \begin{itemize}
                            \item Resolver operatoria de números enteros.
                            \item Utilizar la operatoria de números enteros para resolver problemas contextualizados.
                            \item Resolver operatoria con fracciones y decimales positivos.
                            \item Resolver problemas que involucren el uso de fracciones y decimales.
                        \end{itemize}
                    }}}}
    \vspace{0.2in}
\end{center}




\begin{center}
    \fbox{\fbox{\parbox{16cm}{
                {\textbf{Instrucciones:}
                        \begin{itemize}
                            \item Tienes 80 minutos para contestar esta evaluación.
                            \item La evaluación tiene un total de 30 puntos
                            \item Recuerda que no puedes utilizar dispositivos electrónicos durante la evaluación.
                            \item Lee cada pregunta cuidadosamente. Pon atención a los detalles.

                            \item Esta prueba consta de 30 ppreguntas de seleccioón única. Tal cual lo indica el nombre, solo existe una alternativa correcta a la pregunta o situación planteada. En caso de seleccioar 2 o más, o bien, no seleccionar ninguna, se considerará incorrecto.

                        \end{itemize} }}}}
    \vspace{0.2in}
\end{center}

\newpage
\parbox{5in}{
    {\textsc{\textbf{Preguntas de selección única.} \color{gray}{Habilidad: Conocer, Resolver y Aplicar}}}}

\vspace{0.15in}
\hrule
%\vspace{0.35in}

\begin{questions}

    %\begin{multicols}{2}

    \question[1] ¿Cuál es el resultado de $-10+5+(-8)$?

    \begin{choices}
        \choice $-23$
        \CorrectChoice $-13$
        \choice $18$
        \choice $3$
    \end{choices}

    \vspace{0.15in}

    \question[1] Si $m<0$, ¿cuál de las siguientes expresiones representa un entero negativo?

    \begin{choices}
        \choice $m\cdot m$
        \choice $-m$
        \CorrectChoice $3m$
        \choice $m\div (-2)$
    \end{choices}

    \vspace{0.15in}


    \question[1] El resultado de la expresión $\left[-5+(-3)\cdot 7\right]\div (-2)$ es:
    \begin{choices}
        \choice $24$
        \CorrectChoice $13$
        \choice $-24$
        \choice $-13$
    \end{choices}

    \vspace{0.15in}

    \setlength{\multicolsep}{0.5em}
    \question[1] Si al entero $(-1)$ le restamos el entero $(-4)$, resulta:

    \begin{choices}
        \choice $0$
        \choice $-3$
        \choice $2$
        \CorrectChoice $3$
    \end{choices}
    \vspace{0.15in}


\newpage
    \question[1] Arquímides nació en el año 287 a. C y murió en el año 212 a. C. ¿A que edad murió Arquímides?
    \begin{choices}
        \choice A los 55 años.
        \choice A los 65 años.
        \CorrectChoice A los 75 años.
        \choice A los 85 años.
    \end{choices}
    \vspace{0.15in}

    \question[1] Un comerciante compró al por mayor, 960 lentes de Sol para vender. Si el fabricante de los lentes le regaló uno más por cada docena comprada, ¿cuántos lentes en total recibió el comerciante?
    \begin{choices}
        \choice $961$
        \choice $972$
        \choice $980$
        \CorrectChoice $1.040$
    \end{choices}
    \vspace{0.15in}

    \question[1] ¿Cuál es el resultado de $3\cdot (18-(-7))\div \left[5\cdot (-12+9)\right]$?
    \begin{choices}
        \choice $7$
        \choice $5$
        \CorrectChoice $-5$
        \choice $-7$
    \end{choices}
    \vspace{0.15in}

    \question[1] En la recta numérica, ¿cuál de los siguientes números está a la izquierda de $-17$?
    \begin{choices}
        \CorrectChoice $-18$
        \choice $-16$
        \choice $-0$
        \choice $17$
    \end{choices}
    \vspace{0.15in}

    \question[1] ¿Cuál de las siguientes frases \textbf{NO} se relaciona con el número $-21$?
    \begin{choices}
        \choice Un buzo está a 21 m de profundidad.
        \CorrectChoice La distancia de dos edificios es 21 m.
        \choice La temperatura es 21 °C bajo cero.
        \choice El personaje nació 21 años a.C.
    \end{choices}
    \vspace{0.15in}

\newpage
    \question[1] ¿Qué grupo de números está correctamente ordenado?
    \begin{choices}
        \choice $-15<-12<7<3$
        \choice $7>3>-15>-12$
        \choice $3<7<-12<-15$
        \CorrectChoice $7>3>-12>-15$
    \end{choices}
    \vspace{0.15in}

    \question[1] Al resolver $\left(\dfrac{5}{9}-\dfrac{2}{5}\right)\div \dfrac{14}{15}$ se obtiene
    \begin{choices}
        \choice $\dfrac{1}{14}$
        \choice $\dfrac{45}{56}$
        \choice $\dfrac{98}{975}$
        \CorrectChoice $\dfrac{1}{6}$
    \end{choices}
    \vspace{0.15in}

    \question[1] $(0,8\div 0,2)\div 0,02$
    \begin{choices}
        \choice $\dfrac{1}{200}$
        \choice $\dfrac{1}{8}$
        \choice $0,02$
        \CorrectChoice $200$
    \end{choices}
    \vspace{0.15in}


    \question[1] $\frac{1}{5}+\dfrac{1}{5}+\dfrac{1}{5}+\dfrac{1}{5}$
    \begin{choices}
        \choice $\dfrac{1}{625}$
        \choice $\dfrac{4}{625}$
        \choice $\dfrac{4}{20}$
        \CorrectChoice $\dfrac{4}{5}$
    \end{choices}
    \vspace{0.15in}

\newpage
    \question[1] En una competencia de natación, Anita, Cata y Maca demoraron $25,4$ segundos, $25,03$ segundos y $25,3$ segundos en llegar a la meta, respectivamente. ¿Cuál de las siguientes afirmaciones es \textbf{VERDADERA}?
    \begin{choices}
        \choice Anita llegó antes de Maca.
        \choice Maca llego 27 centesimas antes de Cata.
        \CorrectChoice Cata llegó primera.
        \choice Anita ganó la competencia.
    \end{choices}
    \vspace{0.15in}

    \question[1] Si al cociente entre $\dfrac{6}{12}$ y $\dfrac{2}{14}$ se le resta $\dfrac{15}{10}$ se obtiene:

    \begin{choices}
        \choice $\dfrac{10}{7}$
        \choice $\dfrac{7}{19}$
        \choice $1$
        \CorrectChoice $2$
    \end{choices}

    \question[1] ¿Cuánto se obtiene si el producto $0,5\cdot 0,05$ se divide por el producto $2,5\cdot 0,025$?
    \begin{choices}
        \choice $0,04$
        \CorrectChoice $0,4$
        \choice $2,5$
        \choice $25$
    \end{choices}

    \question[1] ¿Cuál de las siguientes es la representación en notación decimal de la fracción $\dfrac{105}{11}$?
    \begin{choices}
        \choice $9,54$
        \choice $9,5\overline{4}$
        \CorrectChoice $9,\overline{54}$
        \choice $9,54\overline{5}$
    \end{choices}
    \vspace{0.15in}

    \question[1] ¿Cuál de las siguientes es la representación en notación decimal de la fracción $\dfrac{7}{40}$?
    \begin{choices}
        \CorrectChoice $0,175$
        \choice $0,17\overline{5}$
        \choice $1,75$
        \choice $1,\overline{75}$
    \end{choices}
    \vspace{0.15in}

\newpage
    \question[1] ¿Cuál de las siguientes es la representación fraccionaria del decimal $1,05$?
    \begin{choices}
        \choice $\dfrac{105}{10}$
        \choice $\dfrac{95}{100}$
        \choice $\dfrac{95}{99}$
        \CorrectChoice $\dfrac{105}{100}$
    \end{choices}
    \vspace{0.15in}

    \question[1] ¿Cuál de las siguientes es la representación fraccionaria del decimal $0,4\overline{9}$?
    \begin{choices}
        \choice $\dfrac{49}{90}$
        \CorrectChoice $\dfrac{45}{90}$
        \choice $\dfrac{49}{99}$
        \choice $\dfrac{49}{100}$
    \end{choices}
    \vspace{0.15in}

    \question[1] ¿Cuál de las siguientes es la representación fraccionaria del decimal $3,\overline{025}$?
    \begin{choices}
        \choice $\dfrac{3025}{100}$
        \choice $\dfrac{3025}{1000}$
        \CorrectChoice $\dfrac{3022}{999}$
        \choice $\dfrac{3022}{9000}$
    \end{choices}
    \vspace{0.15in}

    \question[1] ¿Cuál es el resultado de $9+9\div 9-9\cdot 9$?
    \begin{choices}
        \choice $-79$
        \CorrectChoice $-71$
        \choice $-63$
        \choice $63$
    \end{choices}
    \vspace{0.15in}

\newpage
    
    \question[1] Si en dos bolsas hay 12 panes en cada una y se sacan $\dfrac{3}{4}$ de la primera y $\dfrac{1}{2}$ de la segunda, ¿cuántos panes se sacaron en total?
    \begin{choices}
        \choice 6 panes
        \CorrectChoice 9 panes
        \choice 12 panes
        \choice 15 panes
    \end{choices}
    \vspace{0.15in}

    \question[1] ¿Cuál es el valor de $4\cdot \dfrac{0,004}{0,04}$?
    \begin{choices}
        \choice $0,1$
        \CorrectChoice $0,4$
        \choice $0,\overline{4}$
        \choice $1,6$
    \end{choices}
    \vspace{0.15in}

    \question[1] Una pizza fue cortada en 12 partes iguales y cada parte pesa $0,13$ kg, ¿Cuál es el peso de la pizza completa?
    \begin{choices}
        \choice $0,15600$ kg
        \CorrectChoice $1,5600$ kg
        \choice $15,600$ kg
        \choice $156,00$ kg
    \end{choices}
    \vspace{0.15in}

    \question[1] Joaquín tiene un emprendimiento de maní. Si tiene $2,230$ kg de maní y quiere repartirlos en bolsitas de $0,25$ kg, ¿cuántas bolsitas con el peso correspondiente puede enbolsar como máximo con todo el maní que tiene?
    \begin{choices}
        \choice $7$
        \CorrectChoice $8$
        \choice $9$
        \choice $10$
    \end{choices}
    \vspace{0.15in}

    \question[1] ¿Cuál de las siguientes afirmaciones es \textbf{VERDADERA}?

    \begin{choices}
        \choice Todo número decimal se puede representar como una fracción entre números enteros.
        \choice Toda fracción se puede representar como un número decimal finito.
        \choice Todo número decimal se puede escribir como una fracción de números enteros donde el denominador sea una potencia de 10, es decir, 10, 100, 1000, 10000, etc.
        \CorrectChoice Toda fracción de números enteros se puede representar como un número decimal.
    \end{choices}
    \vspace{0.15in}

    \question[1] ¿Cuál es el resultado de $-1\cdot 1\div (-1)\cdot 1\cdot (-1)$
    \begin{choices}
        \choice $-2$
        \CorrectChoice $-1$
        \choice $0$
        \choice $1$
    \end{choices}
    \vspace{0.15in}

\end{questions}
\end{document}