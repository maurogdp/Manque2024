\documentclass[spanish,letterpaper, 11pt, addpoints, answers]{exam}
\usepackage[left=2cm, right=2cm, top=2cm, bottom=2.5cm]{geometry}
\PassOptionsToPackage{T1}{fontenc} 
    \usepackage{fontenc} 
    \usepackage[utf8]{inputenc}
    % Cargar babel y configurar para español
    \usepackage[spanish, es-tabla, es-noshorthands]{babel}
    \usepackage{lmodern}
    \usepackage{amsfonts}
    \usepackage{multirow}
    \usepackage{hhline}
    \usepackage[none]{hyphenat}

    \usepackage{mdframed} %agrega borde al minipage
    \usepackage{tcolorbox}


    %\usepackage{background} %marca de agua

%     \backgroundsetup{
%     scale=8,
%     angle=45,
%     opacity=0.5,
%     color=black,
%     contents={\textcolor{gray!50}{MAURO DÍAZ}}
% }



\usepackage{array}
\usepackage[utf8]{inputenc}
\usepackage{graphics}
\usepackage{color}
\usepackage{amssymb}
\usepackage{amsmath}
\usepackage{enumitem}
\usepackage{xcolor}
\usepackage{cancel}
\usepackage{ragged2e}
\usepackage{graphicx}
\usepackage{multicol}
\usepackage{color}
\usepackage{tikz}
\pointpoints{Punto}{Puntos}

\usepackage{graphicx}
\usepackage{tikz}
\usetikzlibrary{babel,arrows.meta,decorations.pathmorphing, backgrounds,positioning,fit,petri, shapes, shadows}

\usepackage{tikz,color}
\usepackage{pgf-pie} 

\usepackage{stackengine}
\newcommand\xrowht[2][0]{\addstackgap[.5\dimexpr#2\relax]{\vphantom{#1}}}



\CorrectChoiceEmphasis{\color{red}}
\noprintanswers

\renewcommand{\choiceshook}{%
    \setlength{\leftmargin}{30pt}%
}


\everymath={\displaystyle}
\renewcommand{\choicelabel}{\thechoice)}
\renewcommand{\choicelabel}{ 
  \ifnum\value{choice}>0
    \makebox[1.5cm][r]{\raggedleft \thechoice)}
  \else
   \raggedleft \thechoice)
  \fi
}

\setlength{\multicolsep}{0.6em}

\setlength{\columnseprule}{2pt}

\setlength{\columnsep}{2cm}
%%%%%% ---Comment out to add a header image ----

\begin{document}
%\begin{figure}[t]
%\includegraphics[width=1\textwidth,height=1.2\textheight,keepaspectratio]{header-cufm.png}
%\end{figure}

\begin{center}
\textbf{Números racionales} \\
Guía teórica\\
1er semestre 2024
\end{center}
\extraheadheight{-0.5in}

\runningheadrule \extraheadheight{0.15in}

\vspace{0.15in}
\runningheadrule \extraheadheight{0.14in}

\lhead{\ifcontinuation{Pregunta \ContinuedQuestion\ continua\ldots}{}}
\runningheader{Números racionales}{Guía teórica}{1er semestre 2024}
\runningfooter{}
              {\thepage\ de \numpages}
              {}
\vspace{0.05in}

\nopointsinmargin
\setlength\linefillthickness{0.1pt}
\setlength\answerlinelength{0.1in}
\vspace{0.1in}

\parbox{6in}{
\textbf{Tema:} Fracciones y números decimales positivos}
\vspace{0.15in}
\hrule 

\begin{itemize}
  \item Para representar gráficamente una fracción, se puede elegir un polígono, dividirlo en partes iguales, según el denominador, y pintar las regiones respectivas, según el numerador. Así, la fracción $\dfrac{3}{5}$ se puede representar como:
    \begin{figure}[h]
      \centering
      \begin{tikzpicture} 
        \draw[draw= black,fill=gray!30] (13,-1) rectangle(14,1);
        \draw[draw= black,fill=gray!30] (14,1) rectangle(15,-1);
        \draw[draw= black,fill=none] (15,-1) rectangle(16,1);
        \draw[draw= black,fill=none] (16,1) rectangle(17,-1);
        \draw[draw= black,fill=none] (17,-1) rectangle(18,1);
      \end{tikzpicture}
      
    \end{figure}
  
  \item Para ubicar una fracción en la recta numérica, se divide la unidad o entero en segmentos iguales, según el denominador, y se ubica la fracción, según indica el numerador.
  
  \begin{figure}[h]
    \centering
    \begin{tikzpicture} 
      
      \draw[<->] (-1,0)-- (11,0);
      \draw (0,0.2) -- (0,-0.2);
      \draw (1,0.2) -- (1,-0.2);
      \draw (2,0.2) -- (2,-0.2);
      \draw (3,0.2) -- (3,-0.2);
      \draw (4,0.2) -- (4,-0.2);
      \draw (5,0.2) -- (5,-0.2);
      \draw (6,0.2) -- (6,-0.2);
      \draw (7,0.2) -- (7,-0.2);
      \draw (8,0.2) -- (8,-0.2);
      \draw (9,0.2) -- (9,-0.2);
      \draw (10,0.2) -- (10,-0.2);
      \fill (7,0)  circle[radius=3pt];

      \node[below] at (0,-0.2){$0$};
      \node[below] at (10,-0.2){$1$};
      \node[below] at (7,-0.2){$\dfrac{7}{10}$};

    \end{tikzpicture}
    
  \end{figure}

  
  \item Para representar gráficamente un número decimal se puede elegir un polígono y dividirlo en 10, 100, 1.000, etc., partes iguales, según su parte decimal, y luego, pintar las regiones respectivas, según la parte decimal. En tanto, para ubicar números decimales en la recta numérica, se realiza un proceso similar, pero contando cada lugar.


    

          
  \end{itemize}
    


\parbox{6in}{
\textbf{Actividades propuestas}}
\vspace{0.15in}
\hrule 

\begin{questions}

\question Evalua si las afirmaciones son verdaderas (V) o falsas (F).

\begin{itemize}
  \item[\textbf{a.}] \rule{1cm}{0.4pt} Para representar gráficamente el número decimal $0{,}843$, se divide el entero en 100 partes iguales.
  \item[\textbf{b.}] \rule{1cm}{0.4pt} La fracción $\dfrac{13}{4}$ es posible representarla con 4 cuadrados divididos en 4 partes iguales cada uno y pintado 3 completos y una parte del cuarto cuadrado.
  \item[\textbf{c.}] \rule{1cm}{0.4pt} La parte entera de un número decimal debe ser distinta de 0 para poder ubicar el número en la recta numérica.
  \item[\textbf{d.}] \rule{1cm}{0.4pt} Todos los números decimales deben se representados gráficamente con un cuadrado.
\end{itemize}


\question Escribe el número decimal y la fracción que representa cada situación.

  \begin{itemize}
  \item[\textbf{a.}] Roberto compró un cuarto de kilogramo de queso en el supermercado.
  \item[\textbf{b.}] Daniela quiere comprar dos kilogramos y medio de manzanas y medio kologramo de uvas.
  \item[\textbf{c.}] El teatro fue usado en las tres cuartas partes de su capacidad.
  \item[\textbf{d.}] Luis tiene un quinto de la edad que tiene su padre.
  \item[\textbf{e.}] La estatua de Renato es un metro y 25 centímetros.
  \end{itemize}

  \question Representa gráficamente cada número y ubícalo en la recta numérica.
  \begin{multicols}{2}
  \begin{itemize}
    \item[\textbf{a.}] $1{,}25$
    \item[\textbf{b.}] $\dfrac{2}{7}$
    \item[\textbf{c.}] $0{,}16$
    \item[\textbf{d.}] $1\dfrac{1}{3}$
    \item[\textbf{e.}] $4{,}6$
    \item[\textbf{f.}] $0{,}34$
    \item[\textbf{g.}] $\dfrac{4}{10}$
    \item[\textbf{h.}] $0{,}75$
    \item[\textbf{i.}] $4\dfrac{5}{4}$
    \item[\textbf{j.}] $0{,}96$
    \item[\textbf{k.}] $3\dfrac{1}{5}$
    \item[\textbf{l.}] $2{,}25$
    \end{itemize}
  \end{multicols}

\newpage
  \vspace{0.1in}
  \parbox{6in}{
  \textbf{Tema:} Conversión de decimales a fracciones y viceversa}
  \vspace{0.15in}
  \hrule 
  
  \begin{itemize}
    \item La \textbf{fracción decimal} es aquella que tiene como denominador, potencias de 10, es decir, 10, 100, 1.000, etc.
    \item Un \textbf{número decimal finito} es todo aquel que se puede transformar en una fracción decimal.
    \item Para transfiormar un número decimal finito a fracción, en el númerador se escribe el número sin coma y en el denominador, potencias de 10, dependiendo de la cantidad de cifras decimales que este tenga (Por cada cifra decimal es un cero).
    \item Para transformar una fracción a número decimal es posible dividir el númerador por el denominador o bien, amplificar (o simplificar) la fracción hasta obtener una fracción decimal.
    
  \end{itemize}
  
  \parbox{6in}{
  \textbf{Actividades propuestas}}
  \vspace{0.15in}
  \hrule 


\question Determina si las igualdades son verdaderas (V) o falsas (F).

\begin{multicols}{2}
  


\begin{itemize}
  \item[a.] $\dfrac{3}{4}=0{,}75$
  \item[b.] $12{,}3=\dfrac{123}{100}$
  \item[c.] $0{,}24=\dfrac{6}{25}$
  \item[d.] $\dfrac{16}{50}=3{,}2$
  \item[e.] $29{,}25=\dfrac{117}{4}$
  \item[f.] $\dfrac{8}{10.000}=0{,}008$         
\end{itemize}

\end{multicols}

\question Transformar los números decimales a fracciones irreductibles.

\begin{multicols}{2}

\begin{itemize}
  \item[a.] $1{,}2$
  \item[b.] $0{,}39$
  \item[c.] $73{,}18$
  \item[d.] $24{,}05$
  \item[e.] $0{,}007$
  \item[f.] $1{,}01$
  \item[g.] $17{,}275$
  \item[h.] $3{,}0004$
  \item[i.] $0{,}00001$
  \item[j.] $23{,}25$
\end{itemize}

\end{multicols}

\question Transforma las fracciones a número decimal.

\begin{multicols}{2}

  \begin{itemize}
    \item[a.] $\dfrac{13}{10}$
    \item[b.] $\dfrac{2}{100}$
    \item[c.] $\dfrac{18}{5}$
    \item[d.] $\dfrac{27}{2}$
    \item[e.] $\dfrac{21}{250}$
    \item[f.] $\dfrac{21}{25}$
    \item[g.] $\dfrac{3}{10.000}$
    \item[h.] $\dfrac{34}{20}$
    \item[i.] $\dfrac{49}{16}$
    \item[j.] $\dfrac{7}{14}$
  \end{itemize}
  
  \end{multicols}

\question Transforma los números mixtos a número decimal.

\begin{multicols}{2}

  \begin{itemize}
    \item[a.] $1\dfrac{1}{2}$
    \item[b.] $2\dfrac{23}{100}$
    \item[c.] $5\dfrac{1}{5}$
    \item[d.] $4\dfrac{7}{2}$
    \item[e.] $5\dfrac{12}{25}$
    \item[f.] $6\dfrac{14}{20}$
  \end{itemize}
  
  \end{multicols}

\newpage
  \vspace{0.1in}
  \parbox{6in}{
  \textbf{Tema:} Adición y sustracción de fracciones y decimales positivos}
  \vspace{0.15in}
  \hrule 
  
  \begin{itemize}
    \item Para resolver adiciones y sustracciones de fracciones con igual denominador se conserva el denominador y se suman o restan los numeradores.
  
    \item Para resolver adiciones y sustracciones de fracciones con distinto denominador es posible igualr los denominadores a su mínimo común múltiplo (m.c.m.), amplificando o simplificando las fracciones.
    \item Para resolver adiciones y sustracciones de números decimales es posible igualar la cantidad de cifras decimales y luego sumar o restar, conservando la posición de la coma decimal.

    
  \end{itemize}
  
  \parbox{6in}{
  \textbf{Actividades propuestas}}
  \vspace{0.15in}
  \hrule 

  \question Resuelve las adiciones y sustracciones.

  \begin{multicols}{2}
    
 
  \begin{itemize}
    \item[a.] $\dfrac{15}{4}+\dfrac{7}{4}$
    \item[b.] $0{,}18+0{,}25$
    \item[c.] $\dfrac{3}{10}+\dfrac{1}{2}$
    \item[d.] $\dfrac{9}{2}-\dfrac{1}{6}$
    \item[e.] $\dfrac{2}{5}+\dfrac{6}{5}+\dfrac{4}{30}$
    \item[f.] $2{,}678-0{,}13$
    \item[g.] $\dfrac{12}{5}+\dfrac{7}{2}-\dfrac{13}{15}$
    \item[h.] $\dfrac{19}{3}-\dfrac{5}{4}-\dfrac{1}{6}$
    \item[i.] $5-\dfrac{12}{8}+\dfrac{1}{3}$
    \item[j.] $12{,}326+12{,}1+3{,}409$
    \item[k.] $0{,}0001+0{,}003-0{,}00005$
    \item[l.] $5\dfrac{1}{8}+6-3\dfrac{1}{6}$  
  \end{itemize}
 \end{multicols}

  \question Completar la tabla.

  \begin{center}
    \begin{tabular}{|c|c|c|c|c|}\hline
      \textbf{$a$}&\textbf{$b$}&\textbf{$c$}&\textbf{$a+b+c$}&\textbf{$a-b-c$}\\ \hline \xrowht{25pt}
      $\dfrac{15}{7}$&$\dfrac{9}{4}$&$\dfrac{1}{2}$&&\\ \hline \xrowht{25pt}
      $18{,}01$&$0{,}013$&$10{,}53$&&\\ \hline \xrowht{25pt}
      $\dfrac{12}{5}$&$\dfrac{1}{10}$&$2{,}5$&&\\ \hline \xrowht{25pt}
      $13$&$2{,}726$&$5{,}19$&&\\ \hline \xrowht{25pt}
      $5\dfrac{3}{4}$&$\dfrac{13}{8}$&$2\dfrac{18}{3}$&&\\ \hline 
      
    \end{tabular}
  \end{center}

\question Resuelve las adiciones y sustracciones de fracciones y números decimales.

\begin{multicols}{2}
    
 
  \begin{itemize}
    \item[a.] $0{,}1+\dfrac{3}{5}-\dfrac{11}{2}$
    \item[b.] $19{,}27+\dfrac{15}{8}-5{,}05$
    \item[c.] $\dfrac{6}{5}-0{,}5+1{,}2-\dfrac{1}{6}$
    \item[d.] $\dfrac{56}{15}-0{,}25+0{,}2-\dfrac{1}{6}$
    \item[e.] $21{,}08+17{,}9-\left(\dfrac{6}{5}+\dfrac{12}{10}\right)$
    \item[f.] $0{,}1+\dfrac{1}{10}-\left(\dfrac{1}{100}+0{,}1\right)$
    \item[g.] $\left(0{,}875-\dfrac{1}{8}+0{,}001\right)+\left(2-\dfrac{1}{20}\right)$
    \item[h.] $\left(3{,}5+6\dfrac{5}{4}\right)-\left(2\dfrac{1}{3}-5\dfrac{1}{6}\right)+1{,}8$
  \end{itemize}
 \end{multicols}

\question Resuelve los problemas.

\begin{itemize}
  \item[a.] La masa de un barril es $5{,}25$ kg. Si al llenarlo con agua con agua, su masa aumenta a $35{,}01$ kg:
  \begin{itemize}
    \item ¿Cuál es la masa de agua viciada en el barril?
    \item ¿Cómo se representa con fracciones las masas del barril vacío y con agua?
    \item Si ahora, en lugar de agua, el barril es llenado con $66{,}5$ kg de arena, ¿qué fracción representa la masa total del barril lleno?
  \end{itemize}
  \item[b.] Un ciclista desea completar un circuito de 12 km, en 3 etapas. En la primera etapa recorrió $\dfrac{18}{4}$ km del circuito, en la segunda etapa avanzó $\dfrac{25}{6}$ km y en la tercera etapa recorrió $\dfrac{15}{8}$ km del circuito.
  \begin{itemize}
    \item ¿Cuál es la fracción que representa el total del trayecto recorrido por el ciclista?
    \item ¿Cuánto le falta por terminar el circuito? Expresa tu respuesta como fracción.
  \end{itemize}
  \item[c.] De una piscina, primero se extraen $356{,}7$ litros. Luego, se sacan $188{,}28$ litros y finalmente, se sacan 21 litros más.
  \begin{itemize}
    \item Si en la piscina quedan $89{,}02$ litros, ¿qué cantidad de agua tenía la piscina inicialmente?
    \item Representa como una fracción los litros extraídos en total.
    \item Representa como fracción el volumen de agua inicial que tenía la piscina.
  \end{itemize}
  \item[d.] En el cumpleaños de Martín, Gaspar comió $\dfrac{1}{5}$ de la torta, David comió $\dfrac{3}{8}$ de la torta y Camila comió $\dfrac{1}{10}$ de torta. ¿Qué fracción de torta quedó?
  \item[e.] Luciano vende $\dfrac{1}{4}$ de su terreno, arrienda las cinco sextas partes del resto y lo restante lo destina para cultivar sus verduras. ¿Qué porción del terreno lo destina para cultivar sus verduras?
\end{itemize}

\newpage
  \vspace{0.1in}
  \parbox{6in}{
  \textbf{Tema:} Multiplicación y división de fracciones y decimales positivos}
  \vspace{0.15in}
  \hrule 
  
  \begin{itemize}
    \item Para multiplicar números decimales finitos, es posible resolver la multiplicacion como números enteros (sin considerar la coma) y en el producto escribir la coma, según la cantidad total de cifras decimales en ambos factores.
  
    \item Para dividir números decimales finitos es posible transformar el dividendo y el divisor en números enteros, amplificando por potencias de 10, según la mayor cantidad de cifras decimales de los números.

    \item \textbf{Multiplicación y división de fracciones}
\begin{center}
$\dfrac{a}{b}\cdot\dfrac{c}{d}=\dfrac{ac}{bd}$ \hspace{2cm} $\dfrac{a}{b}\div\dfrac{c}{d}=\dfrac{a}{b}\cdot\dfrac{d}{c}=\dfrac{ad}{bc}$

\end{center}

    
  \end{itemize}
  
  \parbox{6in}{
  \textbf{Actividades propuestas}}
  \vspace{0.15in}
  \hrule 

  \question Resuelve las multiplicaciones y divisiones de fracciones, dejando como resultado una fracción irreductible.

  \begin{multicols}{3}
    
 
  \begin{itemize}
    \item[a.] $\dfrac{6}{10}\cdot\dfrac{20}{12}$
    \item[b.] $\dfrac{8}{9}\div\dfrac{24}{36}$
    \item[c.] $\dfrac{32}{28}\div\dfrac{14}{8}$
    \item[d.] $\dfrac{21}{15}\div\dfrac{70}{35}$
    \item[e.] $\dfrac{1}{27}\div\dfrac{11}{81}$
    \item[f.] $\dfrac{6}{10}\cdot\dfrac{20}{12}$
    \item[g.] $\dfrac{48}{52}\cdot\dfrac{36}{64}$
    \item[h.] $\dfrac{18}{40}\cdot\dfrac{16}{24}$
    \item[i.] $\dfrac{42}{22}\div\dfrac{5}{45}$  
  \end{itemize}
 \end{multicols}

  \question Completar la tabla.

  \begin{center}
    \begin{tabular}{|c|c|c|c|c|}\hline
      \textbf{$a$}&\textbf{$b$}&\textbf{$c$}&\textbf{$a+b+c$}&\textbf{$a-b-c$}\\ \hline \xrowht{25pt}
      $\dfrac{15}{7}$&$\dfrac{9}{4}$&$\dfrac{1}{2}$&&\\ \hline \xrowht{25pt}
      $18{,}01$&$0{,}013$&$10{,}53$&&\\ \hline \xrowht{25pt}
      $\dfrac{12}{5}$&$\dfrac{1}{10}$&$2{,}5$&&\\ \hline \xrowht{25pt}
      $13$&$2{,}726$&$5{,}19$&&\\ \hline \xrowht{25pt}
      $5\dfrac{3}{4}$&$\dfrac{13}{8}$&$2\dfrac{18}{3}$&&\\ \hline 
      
    \end{tabular}
  \end{center}

\question Resuelve las adiciones y sustracciones de fracciones y números decimales.

\begin{multicols}{2}
    
 
  \begin{itemize}
    \item[a.] $0{,}1+\dfrac{3}{5}-\dfrac{11}{2}$
    \item[b.] $19{,}27+\dfrac{15}{8}-5{,}05$
    \item[c.] $\dfrac{6}{5}-0{,}5+1{,}2-\dfrac{1}{6}$
    \item[d.] $\dfrac{56}{15}-0{,}25+0{,}2-\dfrac{1}{6}$
    \item[e.] $21{,}08+17{,}9-\left(\dfrac{6}{5}+\dfrac{12}{10}\right)$
    \item[f.] $0{,}1+\dfrac{1}{10}-\left(\dfrac{1}{100}+0{,}1\right)$
    \item[g.] $\left(0{,}875-\dfrac{1}{8}+0{,}001\right)+\left(2-\dfrac{1}{20}\right)$
    \item[h.] $\left(3{,}5+6\dfrac{5}{4}\right)-\left(2\dfrac{1}{3}-5\dfrac{1}{6}\right)+1{,}8$
  \end{itemize}
 \end{multicols}

\question Resuelve los problemas.

\begin{itemize}
  \item[a.] La masa de un barril es $5{,}25$ kg. Si al llenarlo con agua con agua, su masa aumenta a $35{,}01$ kg:
  \begin{itemize}
    \item ¿Cuál es la masa de agua viciada en el barril?
    \item ¿Cómo se representa con fracciones las masas del barril vacío y con agua?
    \item Si ahora, en lugar de agua, el barril es llenado con $66{,}5$ kg de arena, ¿qué fracción representa la masa total del barril lleno?
  \end{itemize}
  \item[b.] Un ciclista desea completar un circuito de 12 km, en 3 etapas. En la primera etapa recorrió $\dfrac{18}{4}$ km del circuito, en la segunda etapa avanzó $\dfrac{25}{6}$ km y en la tercera etapa recorrió $\dfrac{15}{8}$ km del circuito.
  \begin{itemize}
    \item ¿Cuál es la fracción que representa el total del trayecto recorrido por el ciclista?
    \item ¿Cuánto le falta por terminar el circuito? Expresa tu respuesta como fracción.
  \end{itemize}
  \item[c.] De una piscina, primero se extraen $356{,}7$ litros. Luego, se sacan $188{,}28$ litros y finalmente, se sacan 21 litros más.
  \begin{itemize}
    \item Si en la piscina quedan $89{,}02$ litros, ¿qué cantidad de agua tenía la piscina inicialmente?
    \item Representa como una fracción los litros extraídos en total.
    \item Representa como fracción el volumen de agua inicial que tenía la piscina.
  \end{itemize}
  \item[d.] En el cumpleaños de Martín, Gaspar comió $\dfrac{1}{5}$ de la torta, David comió $\dfrac{3}{8}$ de la torta y Camila comió $\dfrac{1}{10}$ de torta. ¿Qué fracción de torta quedó?
  \item[e.] Luciano vende $\dfrac{1}{4}$ de su terreno, arrienda las cinco sextas partes del resto y lo restante lo destina para cultivar sus verduras. ¿Qué porción del terreno lo destina para cultivar sus verduras?
\end{itemize}
\end{questions}
\end{document}