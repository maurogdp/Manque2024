\documentclass[spanish,letterpaper, 11pt, addpoints, answers]{exam}
\usepackage[left=2cm, right=2cm, top=2cm, bottom=2.5cm]{geometry}
\PassOptionsToPackage{T1}{fontenc} 
    \usepackage{fontenc} 
    \usepackage[utf8]{inputenc}
    % Cargar babel y configurar para español
    \usepackage[spanish, es-tabla, es-noshorthands]{babel}
    \usepackage{lmodern}
    \usepackage{amsfonts}
    \usepackage{multirow}
    \usepackage{hhline}
    \usepackage[none]{hyphenat}

    \usepackage{mdframed} %agrega borde al minipage
    \usepackage{tcolorbox}


    %\usepackage{background} %marca de agua

%     \backgroundsetup{
%     scale=8,
%     angle=45,
%     opacity=0.5,
%     color=black,
%     contents={\textcolor{gray!50}{MAURO DÍAZ}}
% }



\usepackage{array}
\usepackage[utf8]{inputenc}
\usepackage{graphics}
\usepackage{color}
\usepackage{amssymb}
\usepackage{amsmath}
\usepackage{enumitem}
\usepackage{xcolor}
\usepackage{cancel}
\usepackage{ragged2e}
\usepackage{graphicx}
\usepackage{multicol}
\usepackage{color}
\usepackage{tikz}
\pointpoints{Punto}{Puntos}

\usepackage{graphicx}
\usepackage{tikz}
\usetikzlibrary{babel,arrows.meta,decorations.pathmorphing, backgrounds,positioning,fit,petri, shapes, shadows}

\usepackage{tikz,color}
\usepackage{pgf-pie} 

\usepackage{stackengine}
\newcommand\xrowht[2][0]{\addstackgap[.5\dimexpr#2\relax]{\vphantom{#1}}}



\CorrectChoiceEmphasis{\color{red}}
\noprintanswers

\renewcommand{\choiceshook}{%
    \setlength{\leftmargin}{30pt}%
}


\everymath={\displaystyle}
\renewcommand{\choicelabel}{\thechoice)}
\renewcommand{\choicelabel}{ 
  \ifnum\value{choice}>0
    \makebox[1.5cm][r]{\raggedleft \thechoice)}
  \else
   \raggedleft \thechoice)
  \fi
}

\setlength{\multicolsep}{0.6em}

\setlength{\columnseprule}{2pt}

\setlength{\columnsep}{2cm}
%%%%%% ---Comment out to add a header image ----

\begin{document}
%\begin{figure}[t]
%\includegraphics[width=1\textwidth,height=1.2\textheight,keepaspectratio]{header-cufm.png}
%\end{figure}

\begin{center}
\textbf{Números enteros} \\
Guía teórica\\
1er semestre 2024
\end{center}
\extraheadheight{-0.5in}

\runningheadrule \extraheadheight{0.15in}

\vspace{0.15in}
\runningheadrule \extraheadheight{0.14in}

\lhead{\ifcontinuation{Pregunta \ContinuedQuestion\ continua\ldots}{}}
\runningheader{Números enteros}{Guía teórica}{1er semestre 2024}
\runningfooter{}
              {\thepage\ de \numpages}
              {}
\vspace{0.05in}

\nopointsinmargin
\setlength\linefillthickness{0.1pt}
\setlength\answerlinelength{0.1in}
\vspace{0.1in}
\parbox{6in}{
\textbf{Tema:} El conjunto de los números enteros.}
\vspace{0.15in}
\hrule 

\begin{itemize}
  \item El conjunto de los \textbf{números enteros} ($\mathbb{Z}$) es la unión entre los conjuntos formados por los números naturales, sus inversos aditivos y el cero.
  $$
  \mathbb{Z}=\mathbb{Z^{-}}\cup \{0\}\cup\mathbb{Z^{+}}
  $$
  \item Los \textbf{números enteros positivos} ($\mathbb{Z^+}$) son los números naturales y se pueden escribir con o sin el signo $+$.
  \item Los \textbf{números enteros negativos} ($\mathbb{Z^-}$) son los inversos aditivos de los números naturales y se escriben con el signo $-$.
  \item Cualquier número entero positivo es mayor que 0, mientras que cualquier numero entero negativo es menor que cero.
  \item \textbf{Vaor absoluto:} distancia que existe en la recta numérica entre un número y el cero.
\end{itemize}

\parbox{6in}{
\textbf{Actividades propuestas}}
\vspace{0.15in}
\hrule 

\begin{questions}

\question Completa con los signos $>,<$ o $=$.

\begin{multicols}{2}

\begin{itemize}
  \item[\textbf{a.}] $-5$ \rule{1cm}{0.4pt} $6$
  \item[\textbf{b.}] $0$ \rule{1cm}{0.4pt} $7$
  \item[\textbf{c.}] $|5|$ \rule{1cm}{0.4pt} $-4$
  \item[\textbf{d.}] $-11$ \rule{1cm}{0.4pt} $-12$
  \item[\textbf{e.}] $|-12|$ \rule{1cm}{0.4pt} $-3$
  \item[\textbf{f.}] $|18|$ \rule{1cm}{0.4pt} $|-18|$
  \item[\textbf{g.}] $-15$ \rule{1cm}{0.4pt} $-15$
  \item[\textbf{h.}] $-1$ \rule{1cm}{0.4pt} $-101$
\end{itemize}
\end{multicols}

\question Evalúa si las afirmaciones son verdaderas (V) o falsas (F).

  \begin{itemize}
  \item[\textbf{a.}]\rule{1cm}{0.4pt} El inverso aditivo de $-2$ es $2$.
  \item[\textbf{b.}]\rule{1cm}{0.4pt} El tercer subterraneo de un edificio se puede representar con el número $3$.
  \item[\textbf{c.}]\rule{1cm}{0.4pt} El valor absoluto de un número es su inverso aditivo.
  \item[\textbf{d.}]\rule{1cm}{0.4pt} La distancia entre dos números en la recta numérica siempre se representa con un número entero positivo.
  \item[\textbf{e.}]\rule{1cm}{0.4pt} El termómetro marcó 6 grados Celsius bajo cero, es decir, $-6$ $ ^o$C.
  \item[\textbf{f.}]\rule{1cm}{0.4pt} Al cero no lo antecede un signo porque es positivo.
  \item[\textbf{g.}]\rule{1cm}{0.4pt} Todo número es mayor que su inverso aditivo.
  \item[\textbf{h.}]\rule{1cm}{0.4pt} Todos los números negativos son menores que cualquier número positivo. 
  \end{itemize}

  \question Representa cada situación con un número entero.
  
  \begin{itemize}
    \item[\textbf{a.}]Juan tiene una deuda de \$15.400.
    \item[\textbf{b.}]El submarino llegó hasta 340 metros de profundidad.
    \item[\textbf{c.}]El termómetro registró una temperatura de 18 grados Celsius.
    \item[\textbf{d.}]El colegio está a 300 m de distancia de la casa de Pedro.
    \item[\textbf{e.}]Todos los meses recivo \$25.000.
    \item[\textbf{f.}]La montaña rusa tiene una altua de 85 m.
    \item[\textbf{g.}]El automóvil retrocedio 15 metros.
    \item[\textbf{h.}]Aristóteles nació el año 384 a. C. 
    \end{itemize}

  \question Ubica los números en la recta numérica

  \begin{itemize}
  \item[\textbf{a.}]$-1,2,-4,-6$ y $1$
  \item[\textbf{b.}]$-7,0,-10,-5,-6,5$ y $8$
  \item[\textbf{c.}]$25,-15,5,10,-20$ y $-10$
  \item[\textbf{d.}]$-45,60,-10,-30,20$ y $-35$
  \end{itemize}
  
  \question Ordena los siguientes hechos históricos, de acuerdo al año en que sucedieron.
  
  \begin{itemize}
  \item[\textbf{a.}]\rule{1cm}{0.4pt} En 1492 Cristóbal Colón llegó a América.
  \item[\textbf{b.}]\rule{1cm}{0.4pt} En el año 476 d. C. finalizó la etapa conocida como Edad Antigua.
  \item[\textbf{c.}]\rule{1cm}{0.4pt} La segunda guerra mundial terminó el año 1945.
  \item[\textbf{d.}]\rule{1cm}{0.4pt} La invnción de la escritura data del año 3.000 a. C.
  \item[\textbf{e.}]\rule{1cm}{0.4pt} En el año 1789 se produjo la revolución francesa.
  \item[\textbf{f.}]\rule{1cm}{0.4pt} Hace 3 millones de años a. C. aproximadamente, aparecir la primera especie o forma humana llamada Australopitecos.
  \item[\textbf{g.}]\rule{1cm}{0.4pt} El año 1989 fue la caída del muro de Berlín.
  \item[\textbf{h.}]\rule{1cm}{0.4pt} En el siglo VII a. C. Homero escribio ``La Odisea''.
  \item[\textbf{i.}]\rule{1cm}{0.4pt} Hacia los años 530 a. C. los discípulos de Pitágoras, en Grecia, enseñaron qu el mundo no tenia forma de disco.
  \item[\textbf{j.}]\rule{1cm}{0.4pt} El 14 de marzo del 2018 falleció el científico Stephen Hawking.
  \end{itemize}
\newpage
  \vspace{0.1in}
  \parbox{6in}{
  \textbf{Tema:} Adición y sustracción de números enteros.}
  \vspace{0.15in}
  \hrule 
  
  \begin{itemize}
    \item Para sumar números enteros de igual signo, se suman sus valores absolutos y se conserva el signo de los sumandos.
    \begin{center}
      $3+5=8$\hspace{2cm}$-3+(-5)=-8$
    \end{center}
    \item Para sumar números enteros con distinto signo, se calcula la diferencia de los valores absolutos de los números y se conserva el signo del sumando de mayor valor absoluto.
    \begin{center}
      $-5+2=-3$\hspace{2cm}$6+(-11)=-5$
    \end{center}
    \item Al restar dos números enteros se puede sumar al minuendo el inverso aditivo del sustraendo.
    \begin{center}
      $25-(-4)=25+4=29$\hspace{2cm}$-20-5=-20+(-5)=-25$
    \end{center}
    \item Las adiciones y sustracciones de números enteros también pueden ser resueltas en la recta numérica.
    
  \end{itemize}
  
  \parbox{6in}{
  \textbf{Actividades propuestas}}
  \vspace{0.15in}
  \hrule 


\question Resuelve las siguientes operaciones.

\begin{multicols}{2}
  


\begin{itemize}
  \item[a.] $-7+6$
  \vspace{1cm}
  \item[b.] $22+53$
  \vspace{1cm}
  \item[c.] $34+(-15)$
  \vspace{1cm}
  \item[d.] $-15+16$
  \vspace{1cm}
  \item[e.] $-3+(-56)$
  \vspace{1cm}
  \item[f.] $-17-(-12)$
  \vspace{1cm}
  \item[g.] $-5+(-4)+9$
  \vspace{1cm}
  \item[h.] $-24-(-31)+45$
  \vspace{1cm}
  \item[i.] $56+|-32|-(-19)$
  \vspace{1cm}
  \item[j.] $|-1|+1-(-1)$
  \vspace{1cm}
  \item[k.] $|46|+8-98-|12|$
  \vspace{1cm}
  \item[l.] $19+12+(-27)-|-29|$
  \vspace{1cm}
  \item[m.] $|-23|-|32|+(-78)$
  \vspace{1cm}
  \item[n.] $-67-(-55)+(-70)$
  \vspace{1cm}
  \item[ñ.] $-136-(-234)+|-81|-|-104|$
  \vspace{1cm}
  \item[o.] $-23-(-489)-(-37)-(-48)$
  \vspace{1cm}
  \item[p.] $-28+(-38)-(-93)-(-29)$
  \vspace{1cm}
  \item[q.] $-1+(-1)-(-1)+1+(-1)-(-1)$             
\end{itemize}

\end{multicols}

\question Calcula mentalmente el valor de $x$ para que cumpla la iguadad.

\begin{multicols}{2}

\begin{itemize}
  \item[a.] $x+5=18$
  \item[b.] $x+(-4)=20$
  \item[c.] $3+x=12$
  \item[d.] $x+6=-2$
  \item[e.] $12+x=4$
  \item[f.] $2+x=-17$
  \item[g.] $-23+x=-29$
  \item[h.] $x-(-6)=14$
  \item[i.] $x-(-10)=0$
  \item[j.] $12-x=13$
  \item[k.] $15-x=-15$
  \item[l.] $x-(-32)=32$
  \item[m.] $11=x-9$
  \item[n.] $-15-x=-28$
         
\end{itemize}

\end{multicols}

\question Completa las tablas.

\begin{itemize}
  \item[a.]
\begin{center}
  \begin{tabular}{|c|c|>{\centering\arraybackslash}p{2.5cm}|>{\centering\arraybackslash}p{2.5cm}|>{\centering\arraybackslash}p{2.5cm}|}\hline
    $a$ & $b$ & $a+b$ & $a-b$ & $b-a$ \\ \hline
    $2$ & $6$ &&& \\ \hline
    $-1$ & $7$ &&& \\ \hline
    $4$ & $-8$ &&& \\ \hline
    $-3$ & $-9$ &&& \\ \hline
  \end{tabular}
\end{center}
\vspace{0.5cm}
\item[b.]
\begin{center}
  \begin{tabular}{|c|c|c|>{\centering\arraybackslash}p{3.5cm}|>{\centering\arraybackslash}p{3.5cm}|}\hline
  $x$&$y$&$z$&$x+y-z$&$x-y-z$\\ \hline
  $2$&$-3$&$-1$&&\\ \hline
  $4$&$0$&$-11$&&\\ \hline
  $-5$&$-2$&$-7$&&\\ \hline
  
    
  \end{tabular}
\end{center}
\end{itemize}

\question Resuelve los siguientes problemas.

\begin{itemize}
  \item[a.] La temperatura ambiental de un negocio de productos congelados es de 15 °C, mientras que en el interior del congelador está a 12 °C bajo cero. ¿Cuál es la diferencia entre la temperatura ambintal y la del congelador?
  \vspace{2cm}
  \item[b.] Después de subir 9 pisos, el ascensor de un edificio llegó al quinto piso. ¿Desde qué piso comenzó a subir el ascensor?
  \vspace{2cm}
  \item[c.] Un submariono se encontraba a 140 mtros de profundidad. Si ascendió 70 metros, ¿a qué profundidad llegó?
  \vspace{2cm}
  \item[d.] Pitágoras nació el año 580 a. C. y murió el año 501 a. C. ¿Cuántos años vivió Pitágoras?
  \vspace{2cm}
  \item[e.] En cierto momento, un avión vuela a 2.200 mtros de altura sobre el nivel del mar y un buzo se encuentra a 250 metros de prufundidad del mar. ¿Cuál es la diferencia entre las distancias del avión y el buzo, respecto al nivel del mar?    
  \vspace{2cm}
  \item[f.] Un día de invierno, a las 7:00 hrs el termómetro marcó -3 °C. Luego, a las 13:00 hrs la temperatura ascendió 8 °C respecto a la anteriormente mencionada. Si hasta las 16:00 hrs subió 2 °C, luego, desde las 16 hrs hasta las 23:00 bajó 4 °C y, finalmente, desde las 23:00 hrs hasta las 6 hrs del dia siguiente, bajó 5 °C más, ¿qué temperatura marcó el termometro en la última medición? 
\end{itemize}

\question Resuelve las siguientes adiciones y sustracciones de números enteros.

\begin{itemize}
  \item[a.] $(12+4)-(18)+(-4)$
  \vspace{3cm}
  \item[b.] $|-15+(-13)+6|$
  \vspace{3cm}
  \item[c.] $42+\left[16-(-3)+(-4)\right]$
  \vspace{3cm}
  \item[d.] $\left(-23+(-17)\right)-\left(28-(-10)\right)$
  \vspace{3cm}
  \item[e.] $\left|5-(-19)\right|+\left|-93+(-83)\right|$
  \vspace{3cm}
  \item[f.] $18+\left(28-\left(-78+29\right)-1\right)$
  \vspace{3cm}
  \item[g.] $\left(65-\left|41-42\right|+(-37)-(-11)\right)+(-23)$
  \vspace{3cm}
  \item[h.] $-\left(81+(-71)-(-52)+18\right)$
  \vspace{3cm}
  \item[i.] $\left(24+(-12)\right)-\left|26-(-25)\right|-\left(-32-(-17)\right)$
  \vspace{3cm}
  \item[j.] $-\left(16-\left(-19+(-15)\right)-\left(-1-(-21)-|-32|\right)\right)$       
\end{itemize}

\newpage
  \vspace{0.1in}
  \parbox{6in}{
  \textbf{Tema:} Multiplicación de números enteros.}
  \vspace{0.15in}
  \hrule 
  
  \begin{itemize}
    \item Al multiplicar números enteros de iguales signos, el resultado siempre será positivo.
  
    \item Al multiplicar números enteros de distinto signo, el resultado siempre será negativo.
    \item Las siguientes propiedades se cumplen para todo número entero:
    \begin{itemize}
      \item Conmutatividad: $a\cdot b=b\cdot a$, donde $a$ y $b$ son números enteros.
      \item Asociatividad: $a\cdot (b\cdot c)=(a\cdot b)\cdot c$, donde $a,b$ y $c$ son números enteros.
      \item Clausura: $a\cdot b=k$, donde $a,b$ y $k$ son números enteros.
      \item Distributividad: $a\cdot (b+c)=a\cdot b+a\cdot c$, donde $a,b$ y $c$ son números enteros.
      \item Elemento neutro: $a\cdot 1=a$, donde $a$ es un número entero. 
    \end{itemize}
    
  \end{itemize}
  
  \parbox{6in}{
  \textbf{Actividades propuestas}}
  \vspace{0.15in}
  \hrule 

  \question Resuelve las siguitentes multiplicaciones de números enteros.

  \begin{multicols}{2}
    
 
  \begin{itemize}
    \item[a.] $3\cdot (-4)$
    \vspace{1cm}
    \item[b.] $-5\cdot (-7)$
    \vspace{1cm}
    \item[c.] $18\cdot 2$
    \vspace{1cm}
    \item[d.] $-1\cdot (-19)$
    \vspace{1cm}
    \item[e.] $-12\cdot (-12)$
    \vspace{1cm}
    \item[f.] $-27\cdot 8$
    \vspace{1cm}
    \item[g.] $65\cdot (-4)$
    \vspace{1cm}
    \item[h.] $-73\cdot 6$
    \vspace{1cm}
    \item[i.] $0\cdot (-188)$
    \vspace{1cm}
    \item[j.] $81\cdot 2$
    \vspace{1cm}
    \item[k.] $12\cdot (-5)\cdot 3$
    \vspace{1cm}
    \item[l.] $-1\cdot (-6)\cdot (-1)$
    \vspace{1cm}
    \item[m.] $-1\dot (-1)\cdot (-1)\cdot (-1)$
    \vspace{1cm}
    \item[n.] $-4\cdot (-6)\cdot 10$      
  \end{itemize}
 \end{multicols}

  \question Escribe el factor que falta en las siguientes multiplicaciones de números enteros.

  \begin{multicols}{2}
    
 
    \begin{itemize}
      \item[a.] $12\cdot \rule{1cm}{0.4pt}=-36$
      \item[b.] $\rule{1cm}{0.4pt}\cdot 4=28$
      \item[c.] $\rule{1cm}{0.4pt}\cdot (-2)=-34$
      \item[d.] $-9\cdot \rule{1cm}{0.4pt}=-9$
      \item[e.] $-15\cdot \rule{1cm}{0.4pt}$=60
      \item[f.] $\rule{1cm}{0.4pt}\cdot (-13)=-78$
      \item[g.] $\rule{1cm}{0.4pt}\cdot 24=-72$
      \item[h.] $19\cdot \rule{1cm}{0.4pt}=95$
      \item[i.] $\rule{1cm}{0.4pt}\cdot 11=-77$
      \item[j.] $-13\cdot \rule{1cm}{0.4pt}=169$
      \item[k.] $\rule{1cm}{0.4pt}\cdot (-7)=245$
      \item[l.] $-4\cdot \rule{1cm}{0.4pt}=-100$
       
    \end{itemize}
   \end{multicols}

\question Escribe cada propiedad de la multiplicación que se utilizó en los pasos correspondiente.

\begin{center}
\begin{tabular}{rl}
  $-6\cdot (2+8)+[-4\cdot 1]\cdot (-3)$&$=-6\cdot (2+8)+(-4)\cdot [1\cdot (-3)]$\\
  &$=-6\cdot (2+8)+(-4)\cdot (-3)$\\
  &$=(-6\cdot 2)+(-6\cdot 8)+(-4)\cdot (-3)$\\
  &$=-12+(-48)+(-4)\cdot (-3)$\\
  &$=-12+(-48)+12$\\
  &$=-12+12+(-48)$\\
  &$=-48$\\
\end{tabular}
\end{center}

\question Evalúa si las afirmaciones son verdaderas (V) o falsas (F).

\begin{itemize}
  \item[a.] \rule{1cm}{0.4pt} $a,b\in \mathbb{Z}\Rightarrow a\cdot b>0$ 
  \item[b.] \rule{1cm}{0.4pt} $a,b,c\in \mathbb{Z}\Rightarrow a\cdot b\cdot c \in \mathbb{Z}$ 
  \item[c.] \rule{1cm}{0.4pt} $-1$ es el elemento neutro para la multiplicación en $\mathbb{Z}$ 
  \item[d.] \rule{1cm}{0.4pt} $a<0 \wedge b<0\Rightarrow a\cdot b>0$ 
\end{itemize}

\question Completa las tablas.

\begin{itemize}
  \item[a.]
\begin{center}
  \begin{tabular}{|c|c|>{\centering\arraybackslash}p{2.5cm}|>{\centering\arraybackslash}p{2.5cm}|>{\centering\arraybackslash}p{2.5cm}|}\hline
    $a$ & $b$ & $a\cdot b$ & $a\cdot (-b)$ & $-a\cdot (-b)$ \\ \hline
    $3$ & $2$ &&& \\ \hline
    $-1$ & $1$ &&& \\ \hline
    $-6$ & $-4$ &&& \\ \hline
    $-8$ & $-10$ &&& \\ \hline
  \end{tabular}
\end{center}
\vspace{0.5cm}
\item[b.]
\begin{center}
  \begin{tabular}{|c|c|c|>{\centering\arraybackslash}p{3.5cm}|>{\centering\arraybackslash}p{3.5cm}|}\hline
  $x$&$y$&$z$&$x\cdot (-y)\cdot z$&$x\cdot (y-z)$\\ \hline
  $0$&$-3$&$-1$&&\\ \hline
  $2$&$5$&$-4$&&\\ \hline
  $-8$&$-6$&$1$&&\\ \hline
  
    
  \end{tabular}
\end{center}
\end{itemize}

\question Escribe la operación correspondiente de cada problema y luego resuelve.

\begin{itemize}
  \item[a.] Una mini empresa está pasando por una crisis económica. Cada día tiene una pérdida de \$70.000. ¿Cuánto dinero perdió durante la última semana?
  \vspace{2cm}
  \item[b.]Marcela ahorra cada día \$2.500. ¿Cuánto dinero lleva ahorrado en 20 días?
  \vspace{2cm}
  \item[c.]Daniel compró 5 discos en una tienda de música. si cada disco tenía un valor de \$7.990, ¿cuánto dinero gasto por los discos?
  \vspace{2cm}
  \item[d.]el día más frío del año, desde las 00:00 horas la temperatura descendió 3 °C por hora, hasta las 08:00 hrs. ¿Cuál fue la variación total de temperatura entre dicho lapso de tiempo? 
  \vspace{2cm}
  \item[e.] Debido a un virus respiratorio, en un colegio se contagian, en promedio, 4 estudiantes por día. Al cabo de 15 días, ¿cuántos estudiantes estarán contagiados?
  \vspace{2cm}
  \item[f.]Carla gasta semanalmente \$1.500 en frutas. ¿Cuánto dinero gastará en 4 meses? considera 4 semanas por mes.
  \vspace{2cm}
  \item[g.] Juan tiene un plan de telefonía móvil, con un cargo fijo de \$9.990 y un cobro extra de \$5 por minuto. ¿cuánto dinero debe pagar Juan en total, si se excdió en dos horas y media?
  \vspace{2cm}
  \item[h.] En la biblioteca municipal, por cada día de demora del prestamo de un libro, se cobra una multa de \$400. ¿Cuánto se debe pagar de multa, si un libro se entregó 8 días atrasado?
  \vspace{2cm}
  \item[i.] Un automóvil avanza a una rapidez constante de 70 km/h. ¿Cuál es la distancia recorrida luego de 3horas y media?
  \vspace{2cm}
  \item[j.]El cotavo básico quiere realizar una rifa en el colegio y tiene 180 boletos disponibles. Si cada boleto cuesta \$1.500, ¿cuánto dinero podrían ganar con todas las rifas vendidas? 
  \vspace{2cm}
  \item[k.] Un avión comercial vuela a 3.000 m de altura y por cada minuto, desciende 20 m. ¿Cuál es su altura lugo de media hora de vuelo?
\end{itemize}

\newpage
  \vspace{0.1in}
  \parbox{6in}{
  \textbf{Tema:} División de números enteros.}
  \vspace{0.15in}
  \hrule 
  
  \begin{itemize}
    \item Al dividir números enteros de iguales signos, el resultado siempre será positivo; mientras que, al dividir números enteros de distintos signos, el resiltado siempre será negativo.
  
    \item En la división no es posible establecer las propiedades conmutativa, asociativa, distributiva o clausura, pero sí la de elemento neutro: $a\div 1=a$, donde $a$ es cualquier númro entero.
    \item Inverso multiplicativo: $a\cdot \dfrac{1}{a}=1$.
    \item La regla de signos en la división s igual a la regla de signos de la multiplicación.
    \begin{center}
      \begin{tabular}{ccccc}
        $+$&$\cdot$&$+$&$=$&$+$\\
        $-$&$\cdot$&$-$&$=$&$+$\\
        $+$&$\cdot$&$-$&$=$&$-$\\
        $-$&$\cdot$&$+$&$=$&$-$\\
        
      \end{tabular}
    \end{center}
    
    
  \end{itemize}
  
  \parbox{6in}{
  \textbf{Actividades propuestas}}
  \vspace{0.15in}
  \hrule 

  \question Resuelve las divisiones.
  \begin{multicols}{3}
    \begin{itemize}
      \item[a.] $72\div \rule{1cm}{0.4pt}=-9$
      \item[b.] $-56\div \rule{1cm}{0.4pt}=-8$
      \item[c.] $\rule{1cm}{0.4pt}\div 5=-9$
      \item[d.] $\rule{1cm}{0.4pt}\div 13=-4$
      \item[e.] $46\div \rule{1cm}{0.4pt}=-2$
      \item[f.] $129\div (-3)=\rule{1cm}{0.4pt}$
      \item[g.] $-100\div \rule{1cm}{0.4pt}=10$
      \item[h.] $\rule{1cm}{0.4pt}\div (-532)=0$
      \item[i.] $\rule{1cm}{0.4pt}\div (-17)=-17$

   
    \end{itemize}
    
  \end{multicols}

  \question Evalúa si las afirmaciones son verdaderas (V) o falsas (F).

\begin{itemize}
  \item[a.] \rule{1cm}{0.4pt} Si $a\div b=b\div a=k$, entonces, $k=1$. 
  \item[b.] \rule{1cm}{0.4pt} $\forall a,b,c\in \mathbb{Z}$, con $b\neq 0$, si $a\div b=k$, entonces, $k\in \mathbb{Z}$.
  \item[c.] \rule{1cm}{0.4pt} Si $a,b\in \mathbb{Z}$, con $a<0$ y $b>0$, entonces, $a\div b=k$, con $k>0$.
  \item[d.] \rule{1cm}{0.4pt} Si $a,b\in \mathbb{Z}$, con $b\neq 0$, entonces, $a\div b=\dfrac{a}{b}$.
  \item[e.] \rule{1cm}{0.4pt} $a,b\in \mathbb{Z} \wedge b=0 \Rightarrow a\div b=0$
  \item[f.] \rule{1cm}{0.4pt} Si $a,b,c\in \mathbb{Z}-\{0\}\Rightarrow (a\div b)\div c=a\div (b\div c)$ 
  \item[g.] \rule{1cm}{0.4pt} $a,b\in \mathbb{Z}/a>0 \wedge b\neq 0\Rightarrow -\dfrac{a}{b}>0$
  \item[h.] \rule{1cm}{0.4pt} Si $a,b,c\in \mathbb{Z} \wedge b\neq 0\Rightarrow a\div b=c\Leftrightarrow b\cdot c=a$
\end{itemize}

\question Completa las tablas.

\begin{itemize}
  \item[a.]
\begin{center}
  \begin{tabular}{|c|c|>{\centering\arraybackslash}p{2.5cm}|>{\centering\arraybackslash}p{2.5cm}|}\hline
    $a$ & $b$ & $a\div b$ & $-a\div b$ \\ \hline
    $-8$ & $4$ && \\ \hline
    $27$ & $3$ && \\ \hline
    $-168$ & $-6$ && \\ \hline
    $625$ & $-25$ && \\ \hline
  \end{tabular}
\end{center}
\vspace{0.5cm}
\item[b.]
\begin{center}
  \begin{tabular}{|c|c|>{\centering\arraybackslash}p{3.5cm}|>{\centering\arraybackslash}p{3.5cm}|}\hline
  $x$&$y$&$x\div (-y)$&$-x\div (-y)$\\ \hline
  $-105$&$-5$&&\\ \hline
  $-105$&$36$&&\\ \hline
  $-360$&$-15$&&\\ \hline
  $243$&$-81$&&\\ \hline
  
    
  \end{tabular}
\end{center}
\end{itemize}

\question Relaciona las expresiones de cada columna. Para ello únelas con línea.


\begin{minipage}{0.4\linewidth}
  \begin{tcolorbox}[colback=white]
  \begin{center}
    \begin{tabular}{c}
      La mitad de un número\\\\
      $-x\div y=c$\\\\
      $28\div (-14)$\\\\
      $3\cdot (-12)=-36$\\\\
      $-1\div (-b)=c$\\\\
      $-(-10)\div -(-5)$\\\\
      $m\div n$
      
    \end{tabular}
  \end{center}
\end{tcolorbox}
\end{minipage}
\hfill
\begin{minipage}{0.4\linewidth}
  \begin{tcolorbox}[colback=white]
  \begin{center}
    \begin{tabular}{c}
      $c<0$\\\\
      $2$\\\\
      $c>0$\\\\
      $\dfrac{m}{n}$\\\\
      $a\div 2$\\\\
      $-2$\\\\
      $-36\div 3=-12$
      
    \end{tabular}
  \end{center}
\end{tcolorbox}
\end{minipage}



  \question Resuelve los siguientes problemas. Para ello, escribe la operación y luego resuelvela.
  
    \begin{itemize}
      \item[a.] Se lanza una pelota, verticalmente hacia arriba, alcanzando una altura de 180 m. Si luego del primer bote, la altura alcanzada es la tercera parte de la altura inicial, ¿desde qué altura vuelve a caer la pelota luego del primer bote?
      \vspace{2cm}
      \item[b.] Un submarino alcanzó los 248 metros de profundidad. Si en la mitad de su trayecto se encontró con un arrecife de coral, ¿qué operación matemática se debe realizar para determinar la profundidad del arrecife de coral?
      \vspace{2cm}
      \item[c.] En la semana se registraron las siguientes temperaturas mínimas: $-2$ °C, $-1$ °C, $-4$ °C, $-3$ °C, $-1$ °C, $-2$ °C y $-1$ °C. ¿Cuál es la temperatura mínima promedio registrada en ese periodo?
      \vspace{2cm}
      \item[d.] En una división, el dividendo es $-648$, el cociente es $-27$, y el resto es $0$. ¿Cuál es el divisor?

    \end{itemize}
    
    \newpage
    \vspace{0.1in}
    \parbox{6in}{
    \textbf{Tema:} Operaciones combinadas con números enteros.}
    \vspace{0.15in}
    \hrule 
    
    \begin{itemize}
      \item Para resolver operaciones combinadas se debe respetar el siguiente orden:
      \begin{itemize}
        \item[1°] Parentesis (de interior a exterior).
        \item[2°] Multiplicación y/o división (de izquierda a derecha).
        \item[3°] Adición y/o sustracción (de izquierda a derecha).
      \end{itemize}
    
      \item Para eliminar paréntesis se debe aplicar la propiedad distributiva, es decir:
      $+(a+b)=a+b$ o bien $-(a+b)=-a+(-b)$
      
    \end{itemize}
    
    \parbox{6in}{
    \textbf{Actividades propuestas}}
    \vspace{0.15in}
    \hrule 
  
    \question Resuelve los ejercicios, respentando el orden de las operaciones.
    \begin{multicols}{2}
      \begin{itemize}
        \item[a.] $5+15\div (-3)$
        \vspace{3cm}
        \item[b.] $4\cdot (-9)+(-12)\div (-3)$
        \vspace{3cm}
        \item[c.] $[16\div (-4)+5]-[30\div (-6)\cdot 7]$
        \vspace{3cm}
        \item[d.] $-[-13\cdot (-3)-30]\div (-3)+(-18)$ 
        \vspace{3cm}
        \item[e.] $2\cdot \left\{ 9\cdot (-4)\div (-2)+(-7)\right\}$
        \vspace{3cm}
        \item[f.] $[-1\cdot (-1)\div (-1)]-[1\div 1\cdot (-1)]$
        \vspace{3cm}
        \item[g.] $[5+15\div 3]\div [16-(-3)\cdot (-5)]$
        \vspace{3cm}
        \item[h.] $-[4\cdot (-5+4\cdot 3)]+[32\div (-8)]$
        \vspace{3cm}
        \item[i.] $-6\div 2-[12+(-3)\cdot 4\cdot (-5)]$
        \vspace{3cm}
        \item[j.] $(-9\cdot 4\div 2)\cdot 4-[-(11+(-5))\div 3]$
        \vspace{3cm}
        \item[k.] $-[24\div [12-3\cdot 5+(-9)]-14]$
        \vspace{3cm}
        \item[l.] $24\div 6-[16\cdot 6-10\cdot (8+5)]$   
      \end{itemize}
      
    \end{multicols}

    \question Resuelve los problemas. Para ello, plantea y resuelve las operaciones combinadas correspondientes.

    \begin{itemize}
      \item[a.] María tiene \$15.000 ahorrados en su alcancía y cada semana le dan una mesada de \$1.500, de la que ahorra la mitad. ¿Cuánto dinero tendrá ahorrado al cabo de ocho semanas?
      \vspace{3cm}
      \item[b.] La entrada de un parque de diversiones cuesta \$2.500 y adicionalmente se paga \$500 para poder subir a cada juego. ¿Cuánto dinero se paga por una familia, compusta por los padres y 3 hijos, si los padres solo pagan la entrada; mientras que sus 3 hijos pagaron entrada y cada uno subió a 4 juegos?
      \vspace{3cm}
      \item[c.] Bastián tenía ahorrados \$40.000 y jugando fútbol con otros 3 amigos rompieron el ventanal de una casa. Si el costo por romper el ventanal fue de \$96.000, ¿cuánto dinero le quedará en sus ahorros, despues de compartir el pago de forma equitativa con sus amigos para reponer el ventanal roto?
      \vspace{3cm}
      \item[d.] Isidora trabaja en un restaurante, donde por cada día de trabajo le pagan \$15.000 y además, al final de cada semana se reparten las propinas con tres compañeros de trabajo. ¿Cuánto dinero recibe Isidora si esta semana trabajó 6 días y el monto final de las propinas a repartir entre ella y los compañeros de trabajo es de \$183.000?
    \end{itemize}

\end{questions}
\end{document}